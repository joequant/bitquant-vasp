\section{The regulatory challenge}
The great challenge before us is to create standards which protect the
consumer while allowing for innovations and technology growth, and we
appreciate this opportunity to present to the Hong Kong Securities and
Futures Commission our vision and model for virtual asset regulation,
and how we intend to work with the regulators to create a safe and
vibrant virtual asset industry in Hong Kong.

The model for regulation that we are attempting to work with the
Securities and Futures Commission to create is akin to fire codes,
building codes, and food safety standards for restaurants.  One common
aspect of these types of health and safety regulations have been
designed to allow small businesses to comply with these standards, and
in fact, creating safety codes in fact has the effect of promoting the
growth of small businesses and commerce.  The other common aspect of
health and safety regulation is that many of the standards have been
developed through difficult and painful experiences and mistakes
learned.

In developing these standards, we have made use of the experience and
expertise of persons from the Hong Kong precious metals and gemstone
industry, as well as persons in the money services industry.  In
traditional securities regulation, physical security is not a primary
concern because the assets are not bearer assets.  A criminal who
acquires by force or fraud, bank statements or share certificates does
not have anything of value, whereas virtual assets allow for a criminal
with physical possession of cryptocurrency to liquidate those assets
in the same way that they can liquidate gold or diamonds.  However,
just as it is possible to operate a small gold or diamond trading
business in Hong Kong safely and profitably, we believe it is possible
to create a regulatory system by which one can run a virtual asset
trading platform business, and in this operations and policy manual,
we have described an internal structure that describes how we believe
this can be done.

\section{Technological disruption}
Virtual assets are a disruptive technology and to examine the proper
regulatory system for virtual asset regulation, it is necessary to
consider how virtual assets different from traditional securities.
The basic difference between virtual assets and traditional securities
is that virtual assets allow transfer of value through possession of a
private key.  This creates digital analogues for physical possession.

With gold, jewellery, and paper fiat, it becomes possible to control
assets by physically possessing them and to transfer ownership through
a simple transfer of possession.  This contrasts with traditional
securities, which cannot be transferred through transfer of possession.

Because it is possible to transfer virtual assets by
transfer of possession, it becomes possible to create standardized
software and policies to facilitate these transfers.  In this exchange,
we are using the open-source software system OpenCEX.  The use of open
source software allows for the same software to be used by many users,
therefore drastically reducing the cost of software development, and
allowing exchanges to use a common platform and to standardize
internal processes.  Standardized software and processes therefore
allow for relatively small companies to operate profitably as
exchanges without imposing artificial barriers to entry.

\section{Models for SFC regulation}
The Securities Futures Commission prides itself on maintaining stable and
predictable regulation based on past practices.  Although the
conservative nature of SFC regulation is indeed admirable, for virtual
assets, one runs into the problem that the SFC and Hong Kong financial
regulators regulate different institutions in fundamentally different
ways, and hence it is not obvious which method of regulation is
appropriate for virtual asset service providers.

We note that the basic principle that the SFC has used with different
regulated companies is to work with existing technological barriers to
entry, but not to impose regulatory barriers that do not already
exist.

\subsection{Public regulated monopoly model (Exchange licence)}
One possible model for SFC regulation is the model of the Hong Kong
Exchange.  Initially, the Hong Kong market for stock exchanges consisted
of four exchanges, which were merged in the 1980s and combined with the
Hong Kong Futures Exchange.  HKEx currently maintains an effective
monopoly on securities and futures trading in Hong Kong, and its two
subsidiaries, the Stock Exchange of Hong Kong and the Hong Kong Futures
Exchange Limited, are the only institutions authorized to trade
securities in Hong Kong.

The creation of a regulated monopoly or oligopoly with high barriers
is also present in other industries.  For example, Hong Kong taxicabs
and hot dog stands in New York City are regulated to
prevent excessive competition by creating barriers to entry and
ensuring health and safety and maintaining an orderly market.

However, applying the securities exchange model to virtual assets is
not in the public interest for technological reasons.  HKEx can
maintain a monopoly on securities business because the HK Securities
Clearing Company Limited is the only company which can clear and
settle securities trades in Hong Kong using Hong Kong dollar.  By
contrast, virtual assets have a decentralized clearly and settlement
model by which there is no locus by which a regulated monopoly
requirement can be enforced.  The only way that such a regulated
monopoly could be enforced is by creating a financial wall around Hong
Kong, which would not only violate the Basic Law but also destroy Hong
Kong as an international financial center.

\subsection{Technical standards model (Type 7)}

Another model for SFC regulation is the technical standards model,
which is typified by regulation of automated trading systems.  In type
7 licensing, SFC focuses on insuring that operators maintain high
standards of service and reliability and avoid from trades that create
conflict of interest.  The SFC has only issued approximately thirty
Type 7 licences.

Because Hong Kong securities are limited to clearing and
settlement on the HKEx, any automated trading system must be built on
top of a centralized system, which requires a very high capital expense, 
and creates a high barrier to entry.  By contrast, because virtual assets 
exist on top of a decentralized clearing and settlement system, it is possible
through the use of open-source software to have many systems
operate on top of the clearing and settlement system, creating a very low
barrier to entry.  Furthermore, the fact that the underlying clearing
and settlement system is decentralized introduces redundancy in the system,
as the market can tolerate one exchange doing down.  

Because the ability for the regulators to impose technical standards is 
reduced by the low barrier to entry, and because the need for these standards
to maintain market resiliency is reduced by the decentralized nature of the
exchanges, we do not believe that the type 7 model is appropriate for the
regulation of virtual asset exchanges.


\subsection{Brokerage model (Type 1)}

The most appropriate model which we believe for the exchanges is a
type 1 brokerage model. We note that Hong Kong has issued
approximately 1500 type one licenses and that the barriers to entry to
operate a new brokerage are not excessively high.  Based on the
principle that the SFC should not impose any new barriers to entry
that do not already exist technologically, we believe that the
regulatory philosophy and standards for VASP's should resemble type 1
brokerage licensees, with additional safety standards to take into
account the technological aspects of virtual assets.

\section{Safety standards model of investor protection}

Given that we believe that the primary goal of SFC regulation should be
investor asset protection, and given that we believe that the primary
philosophy of regulation should be a health and safety model, we can
describe how we have designed our systems.

Just as an ethical restaurant has both a business and a moral interest
in making sure that their customers are not ill from their food, we
consider it an essential part of our business to do the right thing,
based on experience.

The design of these safety standards involves learning from the
mistakes of the past.  The historical lessons of virtual asset
regulation can be derived from the following historical incidents

\begin{itemize}
\item FTX
\item  3 Arrow Capital
\item Silicon Valley Bank
\item Lehman Mini-bonds
\item Hong Kong Mercantile Exchange
\item hacks of bitcoin exchanges
  \end{itemize}

Furthermore, in designing a regulatory system, we are particularly
concerned with avoiding a situation where the regulation does not, in fact,
protect the consumer and becomes a hindrance to economic growth.  We
are particularly concerned at avoiding a "licence Raj" and preventing
regulatory capture.  We are extremely concerned that regulations will
have the effect of creating extremely high barriers to entry that will
prevent small businesses from competing in the market and allowing for
only large companies with large amounts of capital to be regulated.
These large companies will then lobby for regulations whose effect is
to prevent other companies from competing.

To avoid these outcomes, we believe that we can and should distil the lessons of these 
incidents to a few basic principles which can be used by a small company to create a 
well run exchange.


\subsection{Client money is not the firm's money}

The main lesson that we believe to have been taught by FTX, 3 Arrow
Capital, and Lehman mini-bonds is the need to have clear ownership.  A
virtual asset trading company has three types of assets and three 
different stakeholders.

\begin{itemize}
\item assets owned by the client
\item assets owned by the business
\item assets owned by the persons running the business
\end{itemize}

In the FTX debacle, it was and still unclear which assets were owned
by clients, which assets were owned by FTX, which assets were owned by
Alameda Capital, and which assets were owned by Sam Bankman-Fried.
Similar confusing in asset ownership occurred in the 3 Arrow Capital
situation and the Lehman Mini-bonds   The result of this confusion is
that the principals in the company then move client money into their
own pocket, and because this is other people's money, they have
the capital to undertake reckless actions with other people's money.

The design of the company is set up so that on day one there is a
clear division between client funds, business fund, and personal
assets, and that there are internal controls to prevent client assets
from being mixed with other forms of assets.  In the company design, we
have set up our systems so that client assets are handled by a
dedicated custody team and that functionalities are separated so that
no one person has the internal ability to misappropriate client funds.

Once we have isolated the client assets from misappropriation and
placed them in a vault which is protected from misappropriation both
by governance and technology, we can then introduce other standards
which control access to the vault.  At this level, we introduce
operational procedures so that AML/KYC and investor suitability
procedures are introduced, but as these measures are ``layered onto''
the basic structure, we can introduce AML/KYC and investor
suitability standards on top of the basic framework.

\section{Conclusion}
In conclusion, we believe that the regulatory standards and practices
can be summarized in two principles

\begin{itemize}
\item it is not your money, it is the client's money.
\item know who your client is
\end{itemize}

The remainder of this manual is an effort to elaborate and
operationalize these two principles.

