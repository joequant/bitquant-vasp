\chapter{Infrastructure}

\ifthenelse{\boolean{bankless}}{
\section{Physical infrastructure}


\subsection{Cash management (bankless)}

* All corporate cash must be physically segregated from none personal
cash

* All movement of corporate funds into an out of physical cash must be
recorded

* Physical cash is used only for corporate operations.  Twofish will
not handle physical cash for clients.

* The location of all cash must be physically inventoried

\subsection{Petty cash}
* The company may hold up to HKD 20k in non-secure facilities and in
form of petty cash.

\subsection{Segreated cash}
* Amounts up to HKD 200k may be held in a safe in a non-secure
location.  

\subsection{Physically secure facility}
* Amounts over HKD 200K must be held in a physically separate secure
location.

\subsection{Cash management}
* The company should hold two weeks operational reserve in the form of
paper fiat.

* The company should hold the remaining in the form of two

* The company should at all times have at least two contacts that can
convert paper cash to and from virtual assets
}{}

\section{Software infrastructure}
The software infrastructure consists of the open-source trading system
OpenCEX authored by Polygant.

\section{Cybersecurity}
\subsection{General principles}
\crossref{VA-2: Question 2.5(b)}
Access is to be granted to users on a need-to-know basis.

\subsection{Outsourcing}
\crossref{VA-1: Question 72}
We do not outsource any infrastructure security management.

\subsection{Network infrastructure}
\crossref{VA-1: Question 60}
\crossref{VA-2: Question 2.5(a)}
\status{To be implemented immediately}
Our OpenCEX application systems are currently operating behind a load balancer,
which hides the location of the server from the internet.  We are in
the process of upgrading and hardening the load balancer into an OpenVPN firewall, which will only allow shell access to the server through a virtual intranet.

All of our systems are themselves set up as docker services, which can
be run independently.

\subsection{User access and identity management}
\crossref{VA-1: Question 64}
\status{Fully implemented}

There are three sets of user access management and identity
verification procedures.
\begin{itemize}
\item Staff access to the trading system
\item Staff access to business records
\item Administrative access to the trading system
\end{itemize}

Staff access to the trading system is performed by staff accounts
combined with a user token and 2fA, which allows access to the admin
facilities through the firewall.  All staff access requires 2fA using
Google authenticator, although the account themselves are set up
internally.

Staff access to business accounts is done through accounts using
Google shared accounts and OAuth.

Technical access to the trading system is allowed only to members of
the technology staff, and not to members of either the operations or
custody staff.  Access is available through shell ssh access using
tokens.  There are several levels of authentication:

\begin{itemize}
\item Technical stuff must connect and authenticate with the OpenVPN
firewall using a user token which is controlled by and can be revoked
by the company.
\item The firewall will then require a second level of authentication to
the application server.
\item To perform any administrative function, the user must then switch
  to the application server user (which is not superuser)
\end{itemize}


\subsection{Application logic}
\crossref{VA-1: Question 61}
\status{To be implemented immediately}
Our OpenCEX application server contains a system of regression tests for the
public API.  We will work with the software developers to go through
the public API and improve test coverage to ensure that the systems
have full test coverage.

\subsection{Patch management process}
\crossref{VA-1: Question 62}
\status{Fully implemented}

Our application server is based on OpenCEX which publishes its
source code on GitHub, and we have a maintenance policy that insures
that our systems are no more than one week behind the source code
published on GitHub.

Our servers run Ubuntu Linux and are patched to include the latest
security releases.

\subsection{Antivirus anti-malware systems}
\crossref{VA-1: Question 63}

We are using Ubuntu Linux on our backend systems, and our business
operations are based on systems such as Google Drive.  We also
encourage our employees to use more secure operating systems, such as
Linux or macOS.

Because our systems are web-based, there is no mechanism by which a
virus can 

\subsection{Intrusion detection/protection system}
\crossref{VA-1: Question 53}
\status{To be implemented upon funding}
%%TODO: Need to upgrade our system

Our systems use open-source software to prevent and detect intrusion.
We will place our systems behind an OpenVPN firewall so that the
trading systems are not connected directly to the internet and will
not be directly accessible through the internet.  We will place open
source intrusion detection software such as snort on both the firewall
and the internal servers.

\subsection{Key storage mechanisms}
\crossref{VA-1: Question 54}
\status{Implementation complete by August}
Upon receipt of funding, we will upgrade our cold wallets to use
FIPS-140-2 level 3 certified hardware wallets and require
multi-signature.


\subsection{Application testing}
\crossref{VA-1: Question 55}
\status{Upon receipt of funding}
We are using the open-source software OpenCEX and before the software is
placed into operations we will run the application regression tests
which will ensure that there are no system leaks in the system.  We
are working with the vendor to insure complete regression test
coverage.

In addition, we are in the process of maintaining the services of
``white-hat hackers'' to perform penetration testing.

\subsection{DDoS}
\crossref{VA-1: Question 56}
\status{Critical: To be implemented immediately}
We intend to prevent DDoS attacks by placing our systems behind
Cloudflare.

\subsection{Regular drills}
\crossref{VA-1: Question 57}
\status{To be implemented upon licensing}
We will conduct a drill for security related incidents at least quarterly.


\subsection{Regular independent assessments}
\crossref{VA-1: Question 58}
\status{To be implemented upon licensing}
We are currently speaking with cybersecurity experts and will have at
least biannual assessments of our systems.


\subsection{Physical security}
\crossref{VA-1: Question 71}
\status{Implemented immediately}
Our trading system uses DigitalOcean as a cloud provider and the
provider is responsible for physical security.


\subsection{Policies}
\begin{itemize}
  \item The location of the exchange server is hidden behind a load
  balancer or cloud server.  The IP address of to allow access is not
  publicly available.
  \item All servers are behind a firewall that blocks all ports other
    than 22, 80, and 443
  \item Direct root access to the exchange server is not allowed, and
    shell access to the exchange server is only through passwordless
    tokens
  \item Access through the cloud server console is used only in
    cases where there is no other means of connecting to the server
  \item fail2ban will be run to lock out excessive logins 
\end{itemize}

\subsection{Administrative access}
Administrative access is allowed only for DevOps work.  No trading is
allowed outside the standard trading interfaces.

\begin{itemize}
\item Direct access to administrative accounts is not allowed.  All
  access is to user accounts, with the user taking admin privileges
  later.
\item Nothing runs on the exchange servers apart from the exchange engine
\item (TODO) There should be a two-step login process by which the
  user logs in first to a proxy server, and then from the proxy server
  undertakes a second login to the exchange server
\end{itemize}

\section{Software deployment process}
\subsection{Open Source only software}

The firm has a policy of only running open-source software.  All
software which is run on production's backend services is publicly
available via the GitHub repository for the company.

\subsection{Logs}
The technology department should maintain a log of incidents and
events.

\subsection{New version deployment}
\crossref{VA-1: Question 65a}
The production server will only use known tagged versions of OpenCEX.

The process for deploying a new version of OpenCEX:
\begin{itemize}
  \item New code deployments should generally be done on the weekend.
  \item Deployments must be signed off by the chief technology officer
    or his delegate,
  \item The deployment officer will examine the source code and verify
    that it is the correct version.
  \item Code is downloaded from OpenCEX main site and merged with
    local changes on GitHub.
  \item Code is deployed on a test server and runs for two to three days.
  \item Before there is a server upgrade, the directory containing the
    server database will be backed up.
  \item When there are no anomalies, the code will upgrade on the
    production server
\end{itemize}

\subsection{Major system changes}
\crossref{VA-1: Question 69}

Our OpenCEX system is based on Django python, and we will perform test
upgrades on our test system before pushing any major changes into
production.

\subsection{Scheduled outages}
\crossref{VA-1: Question 66a}
\status{Fully implemented}
We will have scheduled an outage once every day for thirty minutes
between midnight and 00:30 Hong Kong time.  We also schedule a weekly
upgrade on Sundays between midnight and 2:00 HK time.  Orders before
the outage will be preserved, and orders will be available immediately
upon system reload.

We will make the schedule known months ahead of time.

\subsection{Unscheduled outages}
\crossref{VA-1: Question 66c, 67}
In the case of an unscheduled outage, we will assess the likely duration of
such an outage and communicate this information immediately through
our user telegram channel.


\subsection{Vendor patches}
The servers are operating under Linux Ubuntu.  Vendor upgrades should
be made as soon as possible, but in no event more than one week after
the vendor upgrade is released.

\subsection{Server reboot}
The server should be rebooted from a cold start at least once a week.

\subsection{Security issues}
The firm will work closely with vendors to identify and close security
issues.  Security related bugs will be addressed through standard
``responsible'' disclosure mechanism.

\subsection{Backups}
\crossref{VA-1: Question 68}
\crossref{VA-1: Question 2.5(h)}
\periodic{Daily - Technology - Business records, client and
  transaction database backed up off-line}
\status{Backups are with the cloud}
We perform nightly backups of our trading system and business
records.  Our trading systems are set up as virtual machines, and we can
restore the backup on any virtual system.

\subsection{Cybersecurity Training}
\crossref{VA-2: Question 2.2(c)}
\periodic{Annual - Technology - Cybersecurity awareness training provided to all staff}
Cybersecurity awareness training will be provided to all staff at
least once per year.

\subsection{Periodic reviews}
\crossref{VA-2: Question 2.7 (e)(f)}
\crossref{VA-2: Question 2.8 (a)-(f)}
\periodic{Monthly - Technology - Reliability review}
\periodic{Quarterly - Technology - Capacity review}
\periodic{Semiannually - Technology - Contingency review}

The reliability of the system including downtime and lag will be
performed at least once a month by the technology department, and the
capacity of the system, including lag, will be reviewed at least once a
quarter by the technology department.

The backup systems will be tested at semiannually.  This test will
consist of a simulated business disruption to see how quickly the
system can recover using backups and a backup server.
