\chapter{Risk Management}

Risk is our business - James T. Kirk (Return to Tomorrow)

\section{Risk is our business}
The firm as placed as the front the chapter on risk management.  Risk
is unavoidable and is essential for the type of technological and
social progress that the firm seeks.  The goal of the firm is not to
avoid risk, but to manage risk and embrace risk.  Through financial
technology our firm intended to transfer from those who are unable or
unwilling to bare risk to those who are able and willing to assume
risk in exchange for additional reward.

\section{The risk management process}

The risk management process consists of five steps
\begin{itemize}
\item Identify risks
\item Analyze risks
\item Prioritize risks
\item Mitigate risks
\item Monitor risks
\end{itemize}

Risks should not be seen in isolation but holistically.  Extreme care
should be taking to insure that risks are reduced and properly
transfered and that risk are not simply ignored or passed on to
persons or socialized.

This chapter of the policy management should be reviewed each quarter
by senior management in order to identify and prioritize each risk.
Analysis of risk should include not only impact on the firm but also
risks to financial stability.

\subsection{Liquidity, counterparty, market risks}
\crossref{VA1: Question 50}
\status{Implementation ongoing}

As a financial institution we own a legal, moral, and ethical duty to
our clients and society to safeguard the assets we manage.  The key
way that we will deal with liquidity risks is to make sure that client
liabilities and assets are matched, and that the client is able to
withdraw any assets at any time.  We will not take positions using
client funds with other counterparties or take market positions, hence
we are not exposed to market risks, and any market risks are taken by
the clients.

At a later date, we may extend margin loans to our clients, however
these loans are intended to be fully collaterialized using monte carlo
simulations against historical data to insure that positions will be
automatically closed before the collateral is insufficient to repay
the loan.

We intend to mitigate against misuse of client funds by keeping all
client assets in a segregated account under an associated entity which
is independent of the liabilities of the operating company.

With liquidity, counterparty, and market risks mitigated, the only
remaining risks are cybersecurity and risks due to fraud either from
internal or external sources.  We are mitigating these risk by taking
the governance and cybersecurity measures outlined in other places in
this operations manual.


\subsection{Windup and social impact risk}
Risks should include not only impact to the firm but also risk to
clients that result from a voluntary or involuntary liquidation.  In
case of a windup, the firm should be set up to be able to exit the
market with minimial disruption to clients and the overall market.

\section{External risks}

\subsection{Hacking risks}
\subsubsection{Mitigations}
\begin{itemize}
\item minimize the amounts of assets under custody
\item have mechanisms for rapid detection of suspicious activity
\item have the ability to have an emergency shutdown
  \item keep clear audit trails
\end{itemize}
\subsection{Market risks}

The main mechanism by which the firm will deal with market risk is not
to have mismatched liabilities and assets.  Any client funds held
should be held in the same tokens that the liabilities are created in
and should not be exchanged for other tokens.

\subsection{Denial of service risks}
\section{Internal risks}
\subsection{Loss of key personnel}
\subsection{Rogue senior manager}
\subsection{Software failure}
\subsection{Loss of critical data}

Action item - insure that backups are done and perform quarterly tests
to insure that backups are recoverable

\subsection{Growth risks}

\section{Business risks}
\subsection{Critical vendor risk}
\subsection{Windup risks}

One key risk is what happens to the market and the clients in the
event of either a voluntary or involuntary windup.  It is possible
that business objectives will change, and in case of a windup, it is
the responsiblity of the firm to insure that this is done with
minimial disruption and that all creditors are quickly and promptly
paid.

\section{Insurance}
\crossref{Question 99}
\status{Earliest practical date}

We are currently speaking to possible underwriters regarding insurance
and will set up insurance policies as soon as insurance becomes
avaiable.

