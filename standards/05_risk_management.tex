\chapter{Risk Management}

Risk is our business – James T. Kirk (Return to Tomorrow)



\section{The risk management process}
\crossref{Appendix F1 - Part G (i)-(ii)}
\periodic{Monthly - Management - Risk management review}

The senior management should have a risk management review each month
to assess risks faced by the firm.  The review should go through
recent technological and financial changes in the virtual asset
industry and proactively identify possible risks and deficiencies in
risk management.

The risk management process consists of five steps
\begin{itemize}
\item Identify risks
\item Analyse risks
\item Prioritize risks
\item Mitigate risks
\item Monitor risks
\end{itemize}

Risks should not be seen isolated but holistically.  Extreme care
should be taking to ensure that risks are reduced and properly
transferred, and that risk are not simply ignored or passed on to
individuals or socialized.

This chapter of the policy of management should be reviewed each quarter
by senior management to identify and prioritize each risk.
Analysis of risk should include not only the impact on the firm but also
risks to financial stability.

\subsection{Liquidity, counterparty, market risks}
\crossref{VA1: Question 50}
\status{Implementation ongoing}


As a financial institution, we own a legal, moral, and ethical duty to
our clients and society to safeguard the assets we manage.  The key
way that we will deal with liquidity risks is to make sure that client
liabilities and assets are matched, and that the client can
withdraw any assets at any time.  We will not take positions using
client funds with other counterparties or take market positions. Hence,
we are not exposed to market risks, and any market risks are taken by
the clients.

We may extend margin loans to our clients. However,
these loans are intended to be fully collateralized using Monte Carlo
simulations against historical data to ensure that positions will be
automatically closed before the collateral is insufficient to repay
the loan.

We intend to mitigate against misuse of client funds by keeping all
client assets in a segregated account under an associated entity which
is independent of the liabilities of the operating company.

With liquidity, counterparty, and market risks mitigated, the only
remaining risks are cybersecurity and risks due to fraud either from
internal or external sources.  We are mitigating these risk by taking
the governance and cybersecurity measures outlined in other places in
this operations manual.

\subsection{System related risks}
\crossref{Appendix F1: Part G(iii)}
The responsibility for system related risks is the primary
responsibility of the technology and custody function, and it is
the responsibility of these functions to mitigate possible risks that can
be introduced into the system.
