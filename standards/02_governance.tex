\chapter{Governance and Staffing}

This chapter describe the governance and staffing policies of the
exchange.

\section{Delegation of roles}
\crossref{VA1 Question 3}
\status{Fully implemented}

The firm will not delegate functions involving decision-making and
custody to outside entities.

The functions that will be delegated are as follows:
\begin{itemize}
  \item Cloud and server functions.  Running servers and the hardware
    infrastructure will be delegated to cloud providers.  However, all
    software will be run internally.
 \item Business functions.  The firm will use cloud software-as-a-service providers such
   as Google Drive and GitHub for business operations.
 \item KYC and KYT information.  The firm will rely on outside
   providers to receive information concerning KYC and KYT.  However,
   these vendors will be used only for analysis and information
   provision and the firm will not use the firms to make decisions on
   clients.
\item The firm will rely on outside open source firms for development
  of software.  All software used by the firm will be open-source with
  the firm having full source code access and ability to modify sources.
\end{itemize}

The functions that we will not delegate are as follows:
\begin{itemize}
\item Custody of funds
  \item Operation of exchange software.  Our relationship with our
    software vendors is one of open source, by which we receive source
    code from the open source vendors, but all operation is done by
    the firm.
 \item AML/KYC decisions or account management.  While we will rely on
   outside vendors to receive information on AML/KYC the decisions as
   to opening accounts will be made internally
\end{itemize}

The business operations outsourcing will consists of month-to-month
contracts.  Business operations will be outsourced to industry
standard entities for which no special due dilligence is necessary
(i. e. Google cloud or Github).

In cases where there are regulatory issues (i. e. storage of
confidential data) we intended to work with local industry
associations and other exchanges using the same vendors to ensure that
we are able to comply with regulatory requirements.

\section{Associated entity and client funds}
All client funds are to be under the legal ownership of an associated
entity.  The associated entity will maintain separate accounts, and
funds and wallets associated with the associated entity will be kept
separate from the operating business.

The legal form of the associated entity will be that of a whole owned
Hong Kong corporation.  The articles of incorporation will be modified
so that the directors of the company do not have the legal authority
to enter into legal transactions outside the custody of client
funds.  The associated entity will maintain a separate balance sheet
and the funds will not be consolidated with the operating company.

The form of the associated entity is derived from US banking practice
and is intended to maximize the speed of resolution.  In cases where a
bank as failed in the US (i. e. Silicon Valley Bank), the regulators
have been able to quickly seize the legal entity with customer
accounts and then move them over to another operator (i. e. FDIC
seizure of Washington Mutual).

We considered the use of a trust for a legal entity, but while a trust
may provide more investor protections, changing the trustee in
situations where the operating company has failed would be time
consuming and difficult, whereas creating the associated entity as a
corporation with limitation in the articles of incorporation would
allow the regulator to quickly move client funds to another operator.

Client protection is maximized by limitations in the director's power
in the articles of incorporation.  Should the directors attempt to
create off-book liabilities, these liabilities would be ``ultra
vires'' and would become personal liabilities of the director.

\section{Continuity of operations}
\crossref{VA-1: Question 70}
Our business continuity operations are enforced by a system of
mandatory breaks, by which senior management are given two-week leaves
of absences in which contact is forbidden between management and the
company.  In addition, we will conduct a business continuity drill at
least once every six months.

