\chapter{Introduction}

This operations and policy manual covers the design of
the Virtual Asset Trading Platform's proposed structure, governance,
operations, systems and controls, with a focus on key areas such as
governance and staffing, token admission, custody of virtual assets,
KYC, AML/CFT, market surveillance, risk management, and cybersecurity.

The operations and policy manual will describe the design and
rationale behind the policies and procedures for a virtual asset
trading platform.  Each chapter will contain a set of operational
policies and standards, as well as an evaluation section.  This
evaluation section will be reviewed by an external assessor to
indicate compliance with the chapter in question.

\section{History}

This operations and policy manual arose out of the institution gap
that arose from the institution of licensing for virtual asset service
providers in Hong Kong by the Securities and Futures Commission.  The
primary objective of the SFC within Hong Kong is to protect the
investing public through consumer protection and measures to insure
license stability.

As part of the application for licensing the SFC requires that an
applicant prepare a Phase 1 External Assessor Report which is an
assessment of the virtual asset exchange, and once the applicant
receives an approval in principle, the external assessor will prepare
a report describing the implementation of the measures within the
report.  The format and contents of the Phase 1 EAR Report is
specified in Appendix F of the SFC Public Consultation Document dated
February 2023.

However, this creates a dilemma in that neither the SFC nor the
external assessor is in a position to give a virtual asset service
provider directions on what the provider should do.  The job of the
SFC to focus on outcomes, and the SFC cannot and will not provide
specific technical instructions on what a virtual asset service
provider should do.  The job of the external assessor is to audit and
also to judge outcomes, but again, they are not in a position to
design and implement a system.

Moreover, the mechanism of licensing decreases clarity and
transparency, and increases cost.  Without joint communication, each
virtual asset service provider may be redoing the same practices and
expending duplication of resources.  Furthermore, around
cybersecurity and technology, best practices suggest value in sharing
information and increased public understanding and awareness over the
internal processes of an exchange.

It is with this in mind that we have begun our licensing process by
preparing this operations and policies manual and making it open to
the public.  In particular, we have created a sample exchange which
uses open-source trading systems such as OpenCEX and Humming bot in
order to pool the resources of the internet to create strong and
compliant systems.

We are making these documents available in a Creative Commons license
Attribution (CC:BY) required license.

\section{The External Assessor Review process}

The Phase 1 external assessor review ``EAR'' process will begin with
the development of a policy and procedures manual that will state best
practices for the exchange.  The Policies and Procedures manual will
be developed to address the concerns of the Securities Futures
Commission as stated in Appendix F of the February 2023 consultation
document as well as Questionnaire VA-1 of the SFC licensing
application.

Once the operations manual is completed, the firm will undergo a
self-assessment by which the firm will find issues for improvement.

The phase one EAR report (``EAR1'') and the self-assessment will then
be given to a team of external assessors.  The terms of reference for
the assessors for the phase one report will consist of:

\begin{itemize}
\item Are the standard in place adequate for the exchange
  \item Has the self-assessment identified all the outstanding
    issues and come up with a plan for action?
  \item Are the resources that have been allocated sufficient for the
    plan of action?
\end{itemize}

The EAR1 report will be given to the SFC as part of its licensing
application.  Once the SFC gives approval in principle, the firm will
execute the play in the EAR1 report and the assessment team will
review the progress of the firm for the EAR2 report.

\section{Schedule}
\begin{itemize}
\item 2023 July 15—First draft of operations manual finished
  \item 2023 August 15—Estimated date of completion of phase one
    report and initial submission to SFC
    \item 2024 February 29—Absolute deadline for submission of
      application to SFC
    \item 2024 June 1—End of transitional period
\end{itemize}


\section{Organization}

This manual is organized into the following chapters

\begin{itemize}
\item Mission and objectives—This will describe the purpose and
  philosophy of the firm
  \item Governance and Staffing—This will describe the governance
    structure and human resources of the firm
  \item Development and growth.  This section will describe the
    considerations to be undertaken as the company grows and develops.
    Included in this section will be token admission policies.
  \item Operations—This will describe the operations of the
    business.  This section will include Custody of VA, KYC procedures
  \item Risk management—This section will describe the risk
    management process and the specific risks that have been
    identified for the firm.
  \item Infrastructure—This chapter will describe the technical
    infrastructure of the firm.  Included in this section are the
    cybersecurity requirements of the firm
\end{itemize}

\section{Public comment and review}
The operations manual for the exchange is a living document will be
made public, and the public in general and clients of the firm will be
encouraged to propose suggestions for improvement.

\section{Waivers and modifications}
The operations manual will be reviewed by senior management with
public input at least one per year.  In addition, the managing
director may modify the terms of the operations manual by placing a
notice in writing.

The instructions of the operation are standing instructions, however
if it is necessary to protect the interests of the
clients or other stakeholders, a waiver to the instructions in the
manual may be presented.

\section{Contractual obligation}
Nothing in this manual shall establish a contractual obligation with
any outside parties.  However, the contents of the manual are intended
to be considered for fitness for licensure by the regulator.

