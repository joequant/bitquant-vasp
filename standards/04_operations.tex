\chapter{Operations}

\section{Basic principles}

\begin{itemize}
\item IT IS NOT YOUR MONEY.  IT IS THE CLIENT'S MONEY.
\end{itemize}


\section{Custody of Virtual Assets}
\crossref{Appendix F1: Part C}

\subsection{General principles}
\begin{itemize}
  \item All client assets are to be stored in separate wallets from
    business operations wallets.
  \item No staff is allowed to have direct access to either hot
    wallets or cold wallets.  Hot wallets are to be managed by the
    trading system, while cold wallets are to be managed via
    tamperproof hardware devices
\end{itemize}


\section{Wallet management}
All wallets should be inventoried and each wallet should be desginated
as a business wallet, a client funds wallet.

All business wallets and client funds wallets should be held on
separate devices.  All devices with client wallets should contain only
corporate information.

\begin{figure}
\begin{adjustbox}{center}
\begin{tikzpicture}[node distance=1.5cm, every node/.style={draw, minimum width=1.5cm, minimum height=1cm}]

% Blocks
\node (externaldeposit) {\shortstack{External\\wallet}};
\node[right=2cm of externaldeposit] (block1) {\shortstack{Client\\Deposit\\Wallet}};
\node[right=3cm of block1] (block2) {\shortstack{Accumulator\\Wallet}};
\node[below right=1cm and 3cm of block2] (block3) {\shortstack{Internal\\vault\\wallets}};
\node[below=3cm of block1] (block4) {\shortstack{Keeper\\wallet}};
\node[below=3cm of externaldeposit] (block5) {\shortstack{External\\withdrawal\\wallet}};

% Arrows
\draw[->] (externaldeposit) -- (block1);
\draw[->] (block1) -- node[midway, above, draw=none] {\shortstack{Automatic\\sweep}} (block2);
\draw[->] (block2) -- (block3);
\draw[->] (block3) -- node[midway, above, draw=none] {\shortstack{Manual\\transfer\\by\\custody}} (block4);
\draw[->] (block4) -- node[midway, above, draw=none] {\shortstack{Subject\\to\\operations\\approval}} (block5);

% Dotted box
\node[draw, dotted, fit=(externaldeposit)(block5), inner sep=0.5cm, label=above:{User wallets}] {};
\node[draw, dotted, fit=(block1)(block4), inner sep=0.5cm, label=above:{\shortstack{Hot wallets\\in trading system}}] {};
\node[draw, dotted, fit=(block2)(block3), inner sep=0.5cm, label=above:{\shortstack{Cold hardware\\wallets\\held by custody}}] {};
\


\end{tikzpicture}
\end{adjustbox}
\caption{Wallet structure}
\end{figure}

\subsection{Hot cold wallet management}
\crossref{VA-1: Question 89}
All client assets are moved to cold wallets as quickly as possible,
and the amount in the ``keeper'' withdrawal wallet is limited to
amounts which are subject to immediate withdrawal by the trading
system.  We do not have a numerical limit for the amount subject to
hot wallet but will minimize the amounts available in hot wallets.

\subsection{Hot wallet management}
\crossref{VA-1: Question 90}
\status{Fully implemented}
Our trading system OpenCEX maintains only two sets hot
wallets.  User deposit wallets and the keeper withdrawal wallet.
Funds that are deposited into the user deposit wallets are immediately
swept into the cold storage accumulator wallet.

There is another password protected hot wallet which is used by the
OpenCEX to process withdrawals.  The seed and keys for this wallet is
generated by internal by the OpenCEX system using entropy which is
generated by the system using the hardware ``urandom'' device.  The
keys for the keeper wallet is not normally made available to staff,
but could be accessed via administrative access by technical staff.

In order to prevent unauthorized access of the hot wallet keys, the
operations staff do not have access to the technical infrastructure
and the technical staff do not have access to the withdrawal functions
of the system.  

\subsection{Hot wallet testing}
\crossref{VA-1: Question 91}
\status{Fully implemented}
The hot wallets are under the control of the OpenCEX trading system
whic is run in the virtual environment, and we maintain a separate
instance of the trading system and run regression tests against the
system to insure the reliability of the hot wallet.


\subsection{Client internal segregation}
\crossref{VA-1: Question 97}
\status{Fully implemented}
Generally speaking, client virtual assets will be pooled with other
assets.  Clients will have invididual ``hot wallets'' for deposit, but
the contents of those wallets will be swept into cold storage.

\subsection{Cold wallet standards}
\crossref{Appendix F1(ii)-(iii)}

Cold wallets are to be held to the following standards

\begin{itemize}
  \item All cold storage wallets will be accessed via hardware storage
    modules which have been certified at a security rating of
    FIPS-140 Level III
 \item The keys and wallet addresses are to be generated directly from
   the hardware storage module.
 \item A signature for cold storage will require physical possession
   of the hardware device
 \item The hardware devices are to be kept in the main location of the
   firm and all actions involving cold storage will take place within
   the physical location of the business.  The cold storage hardware
   devices are not to leave the physical location of the business, and
   when not in use they will be kept in a safe in the firm.
 \item Each hardware device used in business operations will be
   password protected by a password known only to the user for that
   device.
 \item There will be two sets of backup devices of each wallet which
   are to be held in physicially separate locations in Hong Kong.
   They will be held in the firms legal counsel and the accounting
   firm used by auditing.  The will be tested once every six months to
   insure that the backups devices are functioning.
\end{itemize}

\subsection{Auditing of wallet contents}
\crossref{Appendix F1: Part C (vi)}

Reconcilations will done on a daily basis, in which the contents of
all wallets are to be audited to insure that no transactions are
unaccounted for.

We will develop software that will provide instantenous alerts and
warnings for unauthorized and unexplained transactions.  

\subsection{Business operations wallet}

* A wallet for business operations can contain up to 250k HKD using a
software device in the form of a mobile device.  This mobile device is
the property of the company and should not contain any

\section{Deposit process}
The deposit process consists of the user sending tokens into the
deposit wallet, and then the system moving tokens from the deposit
wallet into the accumulator wallet.  The trading system will generate
the keys of the deposit wallets, and these keys are stored encrypted
in the database.

The private keys of the accumultator wallet are available only with
the a senior manager, and should not be stored on the trading
platform.  Copies of the private keys of the accumulator should be
stored split in two and stored in the archvial location.

\section{Withdrawal process}

The withdrawal process consists of several steps.
\begin{itemize}
  \item Funds are moved from the cold wallets to the hot wallets and
    made available for withdrawal.  Typically this transfer should be
    made once per day and should consist of 120 percent of the average
    expected delay withdrawal volume.
  \item Withdrawals are approved by a staff member
  \item Withdrawals are then queued by the hot wallet and approve by a manager
\end{itemize}

\subsection{Third party fund deposits and withdrawals}
\crossref{VA-1: Question 75}
\status{Fully implemented}
We do not allow for third party withdrawals or third party fiat
deposits unless the account holder is a licensed finance intermediary.
We will to the best of our ability monitor deposit patterns to insure
that the deposits patterns of virtual assets are consistent with the
information provided by the client.

\subsection{Digital wallets ownership of client}
\crossref{VA-1: Question 77}
\status{Fully implemented}
We believe that it is technologically impossible to insure ownership
of wallets, and the most that can be done is to use our anti-fraud /
CFT measures to insure that the use of the system is consistent with
the KYC information provided by the client.

\subsection{Suspension of deposits and withdrawals}
\crossref{VA-1: Question 78}
\status{Fully implemented}
We note that there is no technological process by which we can suspend
deposits to our wallets.  The most that can be done is to not to
credit deposits to client accounts and to not make virtual assets
available for trading.

We will suspend withdrawals and stop crediting a user account under
the following situations:
\begin{itemize}
\item If we believe that the system as a whole has been compromised
\item If we believe that a user account is being used for purposes
  inconsistent with the documentation provided by the client
\item If we believe that the account is being used for money
  laundering or fraudulent activities
\item If we have received binding legal instructions (i.e. a court order)
\end{itemize}

In cases where there is a general outage, the users will be informed
through our telegram site and the SFC will be informed through
standard escalation channels.

In cases, where there is a specific outage, we will handle those
situations on a case by case basis.

\subsection{Deposit and withdrawal limits}
\crossref{VA-1: Question 79}
\status{Fully implemented}
Deposit and withdrawal limit are set by the chief operations officer.

\section{Know your clients (KYC)}
\crossref{Appendix F1: Part D}

The types of accounts that we expect to open can be classified as:

\begin{itemize}
\item Retail accounts
  \item Corporate stcounts
\item Financial intermediaries
\item Staff accounts
  \end{itemize}

Our main clients are expected

\subsection{Professional investors}
\crossref{VA-1: Question 17}
\status{Fully implemented}
We expect that many of our clients will be licensed intermediaries.
In cases where we have confirmed that a client is a licensed
intermediary from a recognized jurisdicition, we will ask for
self-certification of professional investor status and then inform the
licensing authority that we are onboarding the financial intermediary
as a professional investor.

For the purpose of internal policies, the licensed jurisdictions would
include Hong Kong, Singapore, Dubai, the United States, the United
Kingdom, or the European Union, and licenses will include licenses for
brokerages, money services, banking, and insurance.

In the case where we are dealing with a financial intermediary our
policy will be to focus on due diligence in insuring the identity of
the client and that they are in fact a licensed entity.

In cases where we are dealing with non-licensed entities, we will ask
for a bank statement or other proof of assets.  We expect that many of
our clients may have large assets in the form of virtual assets, and
in this case we will ask for a ``satoshi test'' by which the client
will be asked to make a nominal transfer to confirm control over a
wallet.  We will periodic ask for satoshi tests to confirm control
over wallets.

Given the cost and expense of KYC, we would prefer not to have direct
clients either retail or professional investors but rather to work
through financial intermediaries.  In cases where we are approached by
a high net worth individual we will work with them to set up licensed
Hong Kong entity so that we can work with the entity rather than the
individual and the necessary KYC/AML work to pushed off to the entity.

\subsection{Client knowledge}
\crossref{VA-1: Question 18 a, b, c}
\status{Fully implemented}

The firm wishes to focus on sophisticated investors (such as financial
intermediaries).  In cases where the entity has is a licensed
financial intermediary, we will take the existence of a license as
evidence that the client as sufficient knowledge of relevant risks.

We will not take on clients that have no knowledge of virtual assets.

In cases of retail investors, we will ask the individual about their
level of knowledge, and only admit individuals which we believe are
aware of the risks of investing.  We will encourage unsophisticated
investors to go to other exchanges.

The reason this is viable for our firm involves are business model.
Our business model is focused not at retail investors as these
investors require much hand holding and customer support.  Rather by
focusing on active developers and financial intermediaries we
drastically reduce our customer support costs.

In case of a client with no knowledge of virtual assets we would deny
the account and provide references to other exchanges or industry
groups which could provide the necessary training.

\subsection{Financial losses}
\crossref{VA-1: Question 19}
\status{Fully implemented}

The clients on our exchange are focused on sophisicated clients who
are either financial intermediaries or technology sophisicated
developers who are actively in the cryptocurrency business.  We will
not accept accounts from persons that we do not believe to be
sophisticated investors or who we believe to lack necessary financial
resources.

\subsection{Client onboarding}
\crossref{VA-1: Question 20}
\status{Implementation in progress}

We are using the identity provider subsum to identify the identity of
the client.  For Hong Kong retail clients and financial
intermediaries, we will require that the client come to our office for
a face to face meeting, and we will check the licensing status of the
financial intermediary.

\subsection{Client identity}
\crossref{VA-1: Question 21}
We will ask the client to provide standard identity and KYC documents
and then use a third party identity service (Subsum) to verify the
validity of these documents.  In situations where we are dealing with
a financial intermediary, we will require that a representative of the
financial intermediary located in Hong Kong physically meet with us in
our office, and we will verify the licensing status of the financial
intermediary with the licensing authority.

\subsection{Retail clients risk tolerance}
\crossref{VA-1: Question 22a, 22b}
\status{Questionaire to be developed}

For Hong Kong retail clients, we will ask that retail investor fill
out a questionaire and understand the risks and losses that are
possible with virtual assets.  In developing this questionaire we will
work with industry groups in Hong Kong to develop a standardized
questionaire.

As part of our onboarding process, we will ask that all clients reveal
the name of a client reference for whom we can share client and
trading information with.  We expect that in most cases, the client
reference will be a close relative, and in cases where we believe that
the client is undergoing excessive losses we will contact the client
reference.

Generally speaking we will not set risk limits for retail clients.  In
situations where we do not feel that the retail client has sufficient
knowledge to under take the risks of virtual assets, we will not open
an account.  As part of our AML/KYC/Anti-Fraud measures, we will
monitor client accounts for trading losses, and if we find a client
with unusually large losses, we will investigate to seek appropriate
measures.

\subsection{Client identity address and contact details}
\crossref{VA-1: Question 23}
\status{Questionaire to be developed. Scheduled to be developed before
December 2023}
During our client onboarding process, we will ask the client for the
expected volume and type of trades and whether the client is a
financial intermediary.  In cases where the client is expected to have
a large volume of trades or a financial intermediary, we will arrange
for a face to face or video conference call.  We will ask that the
financial intermediary to the call from their office and ask for the
intermediary to video the location of their office.  We will also make
on-site visits to the client office.

\subsection{Use of system and technologies testing}
\crossref{VA-1: Question 24}
We have not yet conducted testing of facial recognition and checking
of security features, but we intend to do spot checking of these
features before becoming fully licensed.

\subsection{Non-face to face channel}
\crossref{VA-1: Question 25}
For non face-to-face channel we intend to make heavy use of video and
face to face meetings in order to verify the identity of the
prospective client.  In particular, with accounts for financial
intermediaries we will require a face to face meeting in the office of
the intermediary.

\subsection{Client master database entry}
\crossref{VA-1: Question 26}
We intend to keep client particulars offline with paper documents
stored in our main office.  Client particulars will not be kept online
and will not be available to trading and operations staff.

\subsection{Client review}
\crossref{VA-1: Question 27}
Each client will undergo a quarterly review in which we will examine
their trading record and identify whether or not their trading is
consistent with their stated activities.  We will also review each
client if unusual trading and withdrawal patterns are noted.

\subsection{Account opening documents}
\crossref{VA-1: Question 28}
We have not completed draft of our account opening documentation, and
we expect to have those documents completed end of Q3 2023.

\section{Account policies}

\subsection{Financial intermediaries}
We consider licensed financial intermediaries including money services
operators, family offices, and license brokers to be low risk,
provided that they are licensed in a reputable jurisdiction such as
Hong Kong, Singapore, and Dubai.  It is our expectation that a
licensed financial intermediary would maintain adequate AML/KYC
protections to satisfy their regulators and hence it would be
unnecessary for us to add an additional level of AML/KYC.

Our main concern regarding due dilligence regarding financial
intermediaries is that they are who they say they are, and that they
the information that the provide is accurate.  In onboarding a
financial intermediary, the staff will check to see that the
intermediary is in fact licensed and that the person opening the
account is an authorized agent of the intermediary.  The staff will
monitor the intermediary with ongoing checks to make sure that there
has been no enforcement action taken.

\subsection{No cross-exchange, vostro, nostro accounts}
The company will not maintain an account with another exchange nor
allow another exchange to hold an account with the company, and will
not maintain nostro or vostro accounts.  Our main concern is that
cross exchange accounts will create ``hidden leverage'' in the
financial system and will cause leverage

\subsection{No third party trading for non-FI accounts}
Persons who are not financial intermediaries are not permitted to
trade on the exchange on behalf of third parties.

\subsection{Fund manager accounts}
\crossref{VA-1: Question 98}
Where a client is a fund manager may be either opened in the name of
the fund manager or in the name of the funds, as determined by the
fund manager,

\section{Basic usage / Demo account}
\crossref{VA-1: Question 29}
Please refer to appendix for demonstration account.

\subsection{Technical specification}
\crossref{VA: Question  30a, b, c}
Our technical specification is available at
https://github.com/Polygant/OpenCEX-backend/tree/master/public\_api
and contains all orders types.  The functional documentation does not
include fee structure.

Our technical specification document does not contain all messages,
requests, error codes and status updates.  However, the source code
for our trading engine is open source and this information can be see
from the source code of our trading engine.

We do have a simulation environment which allows allows for client
trading.  This simulation environment is made on request, and is also
used for cybersecurity testing.

We plan to have a dedicated person for technical support and at least
one person available 24/7.  Our systems are open source, and we intend
to hire an outside vendor for development of our trading systems in
addition to second and third level support.

\subsection{Communications with clients}
\crossref{VA-1: Question 31}
\status{To be implemented by October 2023}
We will have a dedicated telegram channel set up to report on trading
suspension issue.

\subsection{Product information}
\crossref{VA-1: Question 32}
\status{To be implemented by October 2023}
Our website will contain a disclosure document on each virtual asset
available for trading.

\subsection{VA-1: Question 33}
Our exchange will not issue separate statements outside of information
available from the exchange.

\subsection{Mobile devices}
All wallet and cash management will be done via mobile devices issued
by the company.  Staff are not to put wallets information on personal
devices.

\subsection{Login and authentication}
\crossref{VA-1: Questions 36, 36b}
\status{Mandatory two factor authentication not implemented}

We authenticate the identities of the users using two factor
authentication or SMS verfication.

Our staff does not have access to password information, and password
information uses standard hashing protocols to insure that a leak in
the database is resistent to direct attack.  The use of two factor
authentication or SMS verification insures that a password leak does
not allow an attack to gain immediate access to an account.

Also withdrawals require e-mail/SMS verification and must be vetted by
internal staff.

\section{Anti money laundering / counter-financing of terrorism /
  anti-fraud (AML/CFT)}
\crossref{Appendix F1: Part E}

\section{General principles}
\crossref{VA-1: Questions 43, 44}
\status{To be implemented upon funding}

Our money laundering / anti-terrorist financing transactions
procedures are integrated into our anti-fraud transaction procedure,
and focuses on monitoring withdrawals to insure that withdrawals are
not being used for illicit purposes.  All withdrawals are manually
vetting for anti-fraud measures, and we will develop code that will
flag suspicious transactions as our integrated anti-fraud measures.

Our AML/CFT/Anti fraud activities will focus on examining and
carefully vetting client withdrawals.  The primary responsibility for
the AML/CFT/Anti fraud activities will be made by the operations
staff.  The operations staff will be given access on a ``need to
know'' basis on the activities expected by each client

\crossref{VA-1: Question 45}

\section{Red flags}
\crossref{VA-1: Question 46}
\status{Fully implemented}

* Client, or the client’s beneficial owner, being a politically
exposed person or the business relationship assessed is of high risk

In situations where we are dealing with a non-financial intermediary
(i.e. a family office) we will identify the beneficial owner and
monitor the account closely for signs of illicit activity.  However,
we would prefer to avoid duplication of resources and wish to maximize
client privacy, and in these situations we would advise the client to
work with a licensed broker or fund who would perform the necessary
AML/KYC and client vetting and to have the financial intermediary
route orders through the exchange.

With respect to politically exposed persons we are particularly
sensitive to anti-corruption efforts in Mainland China and the use of
virtual assets to fund activities which are prejudicial to the
national security of the People's Republic of China.  In cases where
we believe their to be a national security or anti-corruption risk, we
shall immediately inform the appropriate enforcement authorities.  We
will work only with financial intermediaries who we believe have
similiar policies and will not act against the national interests of
the People's Republic of China or against the laws of the Hong Kong
Special Administrative Region.

* Use of proxies, any unverifiable or high risk geographical location,
disposable email address or mobile number, or use of a constantly
changing device to conduct transactions by the client.

We are against ``blacklisting'' geographical locations provided that
those locations are not subject to UN Security Council sanctions and
are legal under the laws of Hong Kong.  Our exchange will encourage
transactions with locations which have been traditionally underserved,
and we believe that it is in the national security and foreign policy
interest including the ``belt and road initiative'' for Hong Kong to
increase economic activity with areas which may be subject to
blacklist by other powers with interests that are contrary to those of
the People's Republic of China.

We do not consider the us of VPN's to be a high risk activiities, and
we will focus on serving areas of the world which are able to receive
banking and financial services.

We note that many of our clients are using these mechanisms to
disguise transactions which may be illegal in other parts of the
world, but which are legal and allowed under the laws of Hong Kong.
Where we have sufficient assurance from the client that their
activities are legal under the laws of Hong Kong but may be used to
circumvent regulations from other jurisdictions (particularly
jurisdictions with objectives which are contrary to the national
interest of the People's Republic of China) we will allow and in fact
encourage the use of these mechanisms.

* Client transactions involving tainted wallet addresses such as
“darknet” marketplace transactions or tumblers

* Client transactions involving virtual assets with a higher risk or
greater anonymity (for example, virtual assets which mask users’
identities or transaction details)

* Client transactions that are complex, unusually large in amount or
of unusual pattern, or have no apparent economic or lawful purpose

We will examine the underlying motivation behind these transactions
and take appropriate actions based on whether the underlying
motivation is consistent or opposed to the national interests of the
People's Republic of China and the laws of the Hong Kong Special
Administrative Region.  In situations where we conclude that the
underlying activity is lawful under the laws of the HKSAR and
consistent with the national interests of the People's Republic of
China, we will encourage these activities.

\section{Trading and settlment}
Our software infrastructure uses OpenCEX.


\subsection{Order books arrangement}
\crossref{VA-1: Question 37}
\status{Fully implemented}



\subsection{Trade confirmation and post-trade settlement}
\crossref{VA-1: Question 40}
\status{Fully implemented}

Trade confirmations will be performed by sending e-mail to the account
holder.  Trades will settle immediately and there are will be no
process necessary for post-trade settlement.

\subsubsection{No off-platform trading arrangements}
\crossref{VA-1: Question 39 a,b}
\status{Fully implemented}
The exchange will not conduct trading off platform.  All trades for
the exchange will be entered into our platform system.

\subsubsection{No telephone instructions}
\crossref{VA-1: Question 38}
\status{Fully implemented}
The exchange will not conduct or accept instructions via telephone.

\section{Trading rules}

\subsection{Conflict of interests}
\crossref{VA-1: Question 41}
\status{Fully implemented}
The firm itself will not invest in virtual assets or take proprietary
trading positions, will not be a counterparty to the trades, will not
conduct back-to-back traders and will not exchange in market making
trading of virtual assets.

The firm will allow staff and other associated market makers to
perform these activities, but these activities will be performed using
the personal capital of these entities and under the same terms as
other traders.

\subsection{No proprietary trading}
\crossref{VA-1: Question 1}
\status{Fully implemented}

The firm will not engage in propreitary trading.  The firm will not
use either operating funds or client funds for the purpose of
trading.  All liquidity operations will be done by outside
professional investors will not have any explicit or implicit claims
on the firms balance sheet and will be responsible for their own
profit and losses.

The auditor and responsible officers will be tasked with enforcing
this policy, and all staff will be made aware of the policy against
proprietary trading.

\section{Asset Custody}
\subsection{No trading of client funds}
All client funds must be held in the same form that was deposited by
the client.  The firm is strictly prohibited from using client funds
for the purpose of trading.

\subsection{Monitoring}

The following daily reports should be compiled and made available to
the customer advocate.

\section{Client interactions}
\subsection{No solicitation or recommendations}
\crossref{VA-1: Question 5}
\status{Fully implemented}
The firm will not recommend tokens or engage in soliciations for
tokens.  We do not have any plans to exchange any tokens which would
be considered complex products.

\section{Incident reporting procedure}
\crossref{VA-1: Question 4}
\status{Compliance training materials to be developed and hotline to
  be set up}

Our incident reporting procedure consists of the routing reporting
process, and for hotline incident reporting.  All staff will be
made aware of these procedures.

\subsection{Compliance reporting}
\crossref{VA-1: Question 4(a)}
\status{Compliance training materials to be developed}

For reporting of (a) non-compliance with your licensing conditions or
any applicable law, regulations, codes, guidelines, circulars or
FAQs, etc. we will maintain an internal reporting mechanism annd an
external reporting mechanism.

The firm will maintain an alternative reporting chain for
hotline/whistleblowing reporting.  There are two separate use cases.
The first is one in which the reporter believes that co-workers or
their immediate manager is operating in a non-compliant manner but
they have the confidence in senior management.  In this situation the
company will maintain a hotline to compliance and will prepare
compliance training material as to the internal compliance.

The firm will also maintain an outside tipline, by which staff can
report directly to senior management anonymously.

In situations, where staff does not have confidence in the ability of
senior management to deal with the situation, we will provide
compliance training materials outlining whistleblowing procedures by
which staff can contact the SFC directly and in our on-board materials
we will inform staff on whistle-blowing and legal protections that
they have.

\subsection{Suspicous activity procedure}
\crossref{VA-1: Question 4(b)}
\status{Compliance training materials to be developed}

Staff will be trained to encourage a compliance culture with the
following process
\begin{itemize}
  \item Staff will be given compliance training as to how to identify
    and handle suspicious trading activity.  Staff will also be given
    compliance training concerning ``anti-tipping'' procedures and
    that any suspicious activity should not be raised with the client.
  \item In case, staff believes that there is a compliance issue that
    can be addressed by senior management then will be a hotline that
    will allow the staff to raise the issue with compliance
  \item Staff will also be given training on how to raise an issue
    with the financial regulators or law enforcement in case they
    believe that the issue cannot be resolved within the firm.  Staff
    will be given training as to their legal rights and obligations.
\end{itemize}

The firm will also maintain a robust client education procedure, by
which clients will be informed about the dangers of cryptocurrency and
fraudulent activity and about Hong Kong Police anti-fraud activities
through letters of no consent and Mareva injunctions.

\subsection{Technical incident reporting}
\crossref{VA-1: Question 4(c) and 4(d)}
\status{Compliance training materials to be developed}

For routine reporting, the reports are to be made to the department
head, who will deliver reports to the responsible officer who will
then serve as a single point of contact with the SFC.  This procedure
will be used in situation where there are (c) material failures or
interruptions of trading, accounting, clearing, or settlement systems
or (d) material issues identified by outside experts or consultants.

\subsection{Cybersecurity incident reporting}
\crossref{VA-1: Question 52}

We have not encountered any specific cybersecurity incident, although
our monitoring of firewalls has noted the hostile environment that
system faces as we have detected numerous attempts to break into our
system which are stopped at the firewall level.

The escalation procedure is that all cybersecurity incidents will be
reported directly to the head of the division and that compliance will
be noticed of any incidents.  The technology division will immediately
take steps to mitigate the attack.  Reports will be given to
compliance who will notify the regulators as necessary.


\section{Market surveillance}

As the exchange will not be trading proprietary tokens and tokens will
be traded on a different market, we do not 

\section{Client greivance process}

\section{Fraudulent client transaction requests}
\crossref{VA-1: Questions 93 and 94}
Currently all withdrawals are manually vetted and require the approval
of three staff members.  One to fund the hot wallet.  One to approve
the request and one to appove the withdrawal.

In case of possible fraud we will contact the account holder or the
designated representative of the account holder.  We will also collect
all necessary forsenic information and contact the Hong Kong Police as
necessary.


\section{Client fund segreation}
\crossref{VA-1: Question 95}
\status{Fully implemented}
All client funds will be segregated into separate wallets which are to
be owned by the associated entity.

\ifthenelse{\boolean{bankless}}{
\section{Cash and token management for bank-less exchanges}

\subsection{No cash deposits by clients}
The exchange will not hold any cash by clients or accept any cash
deposits.  All paper fiat held by the company are intended for
internal business purposes only.

\subsection{Paper cash must be physically separated}
All paper cash held by the firm must be held in a secure location and
kept strictly separate from other cash, particularly any person cash
held by staff.  All deposits and withdrawals of cash from the firm
must be records in the ledger.

The company will maintain an inventory of all locations which firm
cash is being kept and will insure that adequate security standards.
See physical infrastructure.
}{
\section{Fiat management}
\crossref{VA-1: Question 96}
\status{Scheduled by end of 2023}
It is our intention that client assets be held in a separate bank
account which is legally owned by the associated entity.
}

\section{Staff policies}

\subsection{Communications}
All corporate communications will be done via corporate chat and
corporate e-mails.

\subsection{Staff general policies}

The firm maintains a policy of diversity and inclusion and a zero
tolerance policy toward harrassment of any form.

The firm maintains a strict no-reliation policy regarding bringing up
compliance issues.

\subsection{Staff vetting}

\subsection{Staff onboarding}

Staff should receive the following documents on onboarding:

\begin{itemize}
  \item Copy of policy manual with sections relevant to staff
    highlighted
\end{itemize}

\subsection{Staff trading}
\crossref{VA-2: Question 42}
\status{Fully implemented}

We encourage staff to trade on our systems in the belief that having
our staff use our one products and services will improve them.
However we will maintain the following principles and policies to
maintain protection with the wider financial community

\begin{itemize}
 \item Staff must provide a written statement as to the nature of
   their trading
  \item Staff are only allowed to trade liquid tokens for which the
    firm does not maintain a proprietary interest
  \item Orders by staff can only be executed using personal funds
  \item Orders by staff will be made only through the standard trading
    interfaces.  Staff are strictly prohibited from given staff
    privileged trading over non-staff accounts.
  \item Staff must sign a waiver stating that in case of loss or
    liquidation of the company, that all liabilities to staff will be
    paid out only when liabilities to other clients are paid out in
    full.  Staff must also sign a two year ``clawback'' provision by
    which any profits made by personal trading are subject to
    ``clawback'' in case the firm folds
\end{itemize}

TODO: NEED TO THINK ABOUT QUESTION 42

\subsection{Staff side businesses}
We encourage staff to have side businesses and other crypto related
businesses.  However, staff must report any side businesses which may
create a conflict of interest to senior management.

\section{Security breach protocol}

\begin{itemize}
\item Shut down all services
  \item make assessment as to whether wallets have been breached.  If
    there is any possibility of a wallet breach.  The client funds should be
    moved from cold storage to backup wallets
\item If the machine has been breached it is to be archived for
  forensic analysis.
\item The last known good backups should be used to
  move to a clean machine
  
\end{itemize}    

\section{Data storage}
\subsection{Record keeping policy}


\subsection{Electronic Records Data Storage}
\crossref{VA1: Question 101}
All of our records will be backed up with a location in Hong Kong, and
a backup of all electronic records will be available at our place of business.
