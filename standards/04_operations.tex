\chapter{Operations}

\section{Basic principles}

\begin{itemize}
\item IT IS NOT YOUR MONEY.  IT IS THE CLIENT'S MONEY.
\end{itemize}

\section{Compliance and escalation policy}

All staff will be given

\section{Mobile devices}
All wallet and cash management will be done via mobile devices issued
by the company.  Staff are not to put wallets information on personal
devices.

\section{Account acceptance}

The types of accounts that we expect to open can be classified as:

\begin{itemize}
\item Retail accounts
  \item Corporate accounts
\item Financial intermediaries
\item Staff accounts
  \end{itemize}

Our main clients are expected

\subsection{Professional investors}
\crossref{VA-1: Question 17}
\status{Fully implemented}
We expect that many of our clients will be licensed intermediaries.
In cases where we have confirmed that a client is a licensed
intermediary from a recognized jurisdicition, we will ask for
self-certification of professional investor status and then inform the
licensing authority that we are onboarding the financial intermediary
as a professional investor.

For the purpose of internal policies, the licensed jurisdictions would
include Hong Kong, Singapore, Dubai, the United States, the United
Kingdom, or the European Union, and licenses will include licenses for
brokerages, money services, banking, and insurance.

In the case where we are dealing with a financial intermediary our
policy will be to focus on due diligence in insuring the identity of
the client and that they are in fact a licensed entity.

In cases where we are dealing with non-licensed entities, we will ask
for a bank statement or other proof of assets.  We expect that many of
our clients may have large assets in the form of virtual assets, and
in this case we will ask for a ``satoshi test'' by which the client
will be asked to make a nominal transfer to confirm control over a
wallet.  We will periodic ask for satoshi tests to confirm control
over wallets.

Given the cost and expense of KYC, we would prefer not to have direct
clients either retail or professional investors but rather to work
through financial intermediaries.  In cases where we are approached by
a high net worth individual we will work with them to set up licensed
Hong Kong entity so that we can work with the entity rather than the
individual and the necessary KYC/AML work to pushed off to the entity.

\subsection{Client knowledge}
\crossref{VA-1: Question 18 a, b, c}
\status{Fully implemented}

The firm wishes to focus on sophisticated investors (such as financial
intermediaries).  In cases where the entity has is a licensed
financial intermediary, we will take the existence of a license as
evidence that the client as sufficient knowledge of relevant risks.

We will not take on clients that have no knowledge of virtual assets.

In cases of retail investors, we will ask the individual about their
level of knowledge, and only admit individuals which we believe are
aware of the risks of investing.  We will encourage unsophisticated
investors to go to other exchanges.

The reason this is viable for our firm involves are business model.
Our business model is focused not at retail investors as these
investors require much hand holding and customer support.  Rather by
focusing on active developers and financial intermediaries we
drastically reduce our customer support costs.

In case of a client with no knowledge of virtual assets we would deny
the account and provide references to other exchanges or industry
groups which could provide the necessary training.

\subsection{Financial losses}
\crossref{VA-1: Question 19}
\status{Fully implemented}


\subsection{KYC}

\subsection{Financial intermediaries}
We consider licensed financial intermediaries including money services
operators, family offices, and license brokers to be low risk,
provided that they are licensed in a reputable jurisdiction such as
Hong Kong, Singapore, and Dubai.  It is our expectation that a
licensed financial intermediary would maintain adequate AML/KYC
protections to satisfy their regulators and hence it would be
unnecessary for us to add an additional level of AML/KYC.

Our main concern regarding due dilligence regarding financial
intermediaries is that they are who they say they are, and that they
the information that the provide is accurate.  In onboarding a
financial intermediary, the staff will check to see that the
intermediary is in fact licensed and that the person opening the
account is an authorized agent of the intermediary.  The staff will
monitor the intermediary with ongoing checks to make sure that there
has been no enforcement action taken.

\subsection{No cross-exchange, vostro, nostro accounts}
The company will not maintain an account with another exchange nor
allow another exchange to hold an account with the company, and will
not maintain nostro or vostro accounts.  Our main concern is that
cross exchange accounts will create ``hidden leverage'' in the
financial system and will cause leverage

\subsection{No third party trading for non-FI accounts}
Persons who are not financial intermediaries are not permitted to
trade on the exchange on behalf of third parties.

\section{Deposit process}
The deposit process consists of the user sending tokens into the
deposit wallet, and then the system moving tokens from the deposit
wallet into the accumulator wallet.  The trading system will generate
the keys of the deposit wallets, and these keys are stored encrypted
in the database.

The private keys of the accumultator wallet are available only with
the a senior manager, and should not be stored on the trading
platform.  Copies of the private keys of the accumulator should be
stored split in two and stored in the archvial location.

\section{Withdrawal process}

The withdrawal process consists of several steps.
\begin{itemize}
  \item Funds are moved from the cold wallets to the hot wallets and
    made available for withdrawal.  Typically this transfer should be
    made once per day and should consist of 120 percent of the average
    expected delay withdrawal volume.
  \item Withdrawals are approved by a staff member
  \item Withdrawals are then queued by the hot wallet and approve by a manager
\end{itemize}

\section{Trading rules}

\subsection{No proprietary trading}
\crossref{VA-1: Question 1}
\status{Fully implemented}

The firm will not engage in propreitary trading.  The firm will not
use either operating funds or client funds for the purpose of
trading.  All liquidity operations will be done by outside
professional investors will not have any explicit or implicit claims
on the firms balance sheet and will be responsible for their own
profit and losses.

The auditor and responsible officers will be tasked with enforcing
this policy, and all staff will be made aware of the policy against
proprietary trading.

\section{Asset Custody}
\subsection{No trading of client funds}
All client funds must be held in the same form that was deposited by
the client.  The firm is strictly prohibited from using client funds
for the purpose of trading.

\subsection{Monitoring}

The following daily reports should be compiled and made available to
the customer advocate.

\section{Client interactions}
\subsection{No solicitation or recommendations}
\crossref{VA-1: Question 5}
\status{Fully implemented}
The firm will not recommend tokens or engage in soliciations for
tokens.  We do not have any plans to exchange any tokens which would
be considered complex products.

\section{Incident reporting procedure}
\crossref{VA-1: Question 4}
\status{Compliance training materials to be developed and hotline to
  be set up}

Our incident reporting procedure consists of the routing reporting
process, and for hotline incident reporting.  All staff will be
made aware of these procedures.

\subsection{Compliance reporting}
\crossref{VA-1: Question 4(a)}
\status{Compliance training materials to be developed}

For reporting of (a) non-compliance with your licensing conditions or
any applicable law, regulations, codes, guidelines, circulars or
FAQs, etc. we will maintain an internal reporting mechanism annd an
external reporting mechanism.

The firm will maintain an alternative reporting chain for
hotline/whistleblowing reporting.  There are two separate use cases.
The first is one in which the reporter believes that co-workers or
their immediate manager is operating in a non-compliant manner but
they have the confidence in senior management.  In this situation the
company will maintain a hotline to compliance and will prepare
compliance training material as to the internal compliance.

The firm will also maintain an outside tipline, by which staff can
report directly to senior management anonymously.

In situations, where staff does not have confidence in the ability of
senior management to deal with the situation, we will provide
compliance training materials outlining whistleblowing procedures by
which staff can contact the SFC directly and in our on-board materials
we will inform staff on whistle-blowing and legal protections that
they have.

\subsection{Suspicous activity procedure}
\crossref{VA-1: Question 4(b)}
\status{Compliance training materials to be developed}

Staff will be trained to encourage a compliance culture with the
following process
\begin{itemize}
  \item Staff will be given compliance training as to how to identify
    and handle suspicious trading activity.  Staff will also be given
    compliance training concerning ``anti-tipping'' procedures and
    that any suspicious activity should not be raised with the client.
  \item In case, staff believes that there is a compliance issue that
    can be addressed by senior management then will be a hotline that
    will allow the staff to raise the issue with compliance
  \item Staff will also be given training on how to raise an issue
    with the financial regulators or law enforcement in case they
    believe that the issue cannot be resolved within the firm.  Staff
    will be given training as to their legal rights and obligations.
\end{itemize}

The firm will also maintain a robust client education procedure, by
which clients will be informed about the dangers of cryptocurrency and
fraudulent activity and about Hong Kong Police anti-fraud activities
through letters of no consent and Mareva injunctions.

\subsection{Technical incident reporting}
\crossref{VA-1: Question 4(c) and 4(d)}
\status{Compliance training materials to be developed}

For routine reporting, the reports are to be made to the department
head, who will deliver reports to the responsible officer who will
then serve as a single point of contact with the SFC.  This procedure
will be used in situation where there are (c) material failures or
interruptions of trading, accounting, clearing, or settlement systems
or (d) material issues identified by outside experts or consultants.


\section{Market surveillance}

As the exchange will not be trading proprietary tokens and tokens will
be traded on a different market, we do not 

\section{Client greivance process}

\section{Wallet management}
All wallets should be inventoried and each wallet should be desginated
as a business wallet, a client funds wallet.

All business wallets and client funds wallets should be held on
separate devices.  All devices with client wallets should contain only
corporate information.

\subsection{Business operations wallet}

* A wallet for business operations can contain up to 250k HKD using a
software device in the form of a mobile device.  This mobile device is
the property of the company and should not contain any


\subsection{Hot wallets containing client funds}

\subsection{Cold wallets containing client funds}

\ifthenelse{\boolean{bankless}}{
\section{Cash and token management for bank-less exchanges}

\subsection{No cash deposits by clients}
The exchange will not hold any cash by clients or accept any cash
deposits.  All paper fiat held by the company are intended for
internal business purposes only.

\subsection{Paper cash must be physically separated}
All paper cash held by the firm must be held in a secure location and
kept strictly separate from other cash, particularly any person cash
held by staff.  All deposits and withdrawals of cash from the firm
must be records in the ledger.

The company will maintain an inventory of all locations which firm
cash is being kept and will insure that adequate security standards.
See physical infrastructure.
}{}

\section{Staff policies}

\subsection{Communications}
All corporate communications will be done via corporate chat and
corporate e-mails.

\subsection{Staff general policies}

The firm maintains a policy of diversity and inclusion and a zero
tolerance policy toward harrassment of any form.

The firm maintains a strict no-reliation policy regarding bringing up
compliance issues.

\subsection{Staff vetting}

\subsection{Staff onboarding}

Staff should receive the following documents on onboarding:

\begin{itemize}
  \item Copy of policy manual with sections relevant to staff
    highlighted
\end{itemize}

\subsection{Staff trading}

We encourage staff to trade on our systems in the belief that having
our staff use our one products and services will improve them.
However we will maintain the following principles and policies to
maintain protection with the wider financial community

\begin{itemize}
 \item Staff must provide a written statement as to the nature of
   their trading
  \item Staff are only allowed to trade liquid tokens for which the
    firm does not maintain a proprietary interest
  \item Orders by staff can only be executed using personal funds
  \item Orders by staff will be made only through the standard trading
    interfaces.  Staff are strictly prohibited from given staff
    privileged trading over non-staff accounts.
  \item Staff must sign a waiver stating that in case of loss or
    liquidation of the company, that all liabilities to staff will be
    paid out only when liabilities to other clients are paid out in
    full.  Staff must also sign a two year ``clawback'' provision by
    which any profits made by personal trading are subject to
    ``clawback'' in case the firm folds
\end{itemize}


\subsection{Staff side businesses}
We encourage staff to have side businesses and other crypto related
businesses.  However, staff must report any side businesses which may
create a conflict of interest to senior management.

\section{Security breach protocol}

\begin{itemize}
\item Shut down all services
  \item make assessment as to whether wallets have been breached.  If
    there is any possibility of a wallet breach.  The client funds should be
    moved from cold storage to backup wallets
\item If the machine has been breached it is to be archived for
  forensic analysis.
\item The last known good backups should be used to
  move to a clean machine
  
\end{itemize}    
