\documentclass[]{report}
\usepackage{hyperref}
\hypersetup{
    colorlinks=true,
    linkcolor=blue,
    filecolor=magenta,      
    urlcolor=cyan,
    pdftitle={Overleaf Example},
    pdfpagemode=FullScreen,    
    }

\def\firmfullname{Example Exchange Limited}
\def\firmshortname{Example}
\def\firmtag{example}

\title{Phase 1 External Assessor Report}
\author{\firmfullname\\Copyright 2023 – Licensed under Creative
  Commons CC:BY Licence}
\usepackage{fullpage}
\begin{document}
\maketitle

\chapter{Executive Summary}
This report presents the findings of the first-phase assessment
conducted on the design effectiveness of a platform operator. The
assessment covers various areas, including governance and staffing,
token admission, custody of virtual assets (VA), know-your-clients
(KYC), anti-money laundering and counter-financing of terrorism
… (AML/CFT), market surveillance, risk management, and
cybersecurity.

The assessment report highlights that the platform operator is
currently non-compliant with applicable legal and regulatory
requirements in certain areas. However, the report acknowledges that
the company has developed a comprehensive plan to rectify these
non-compliance issues and ensure future compliance.

As part of the assessment, the firm as developed a comprehensive
policy and operations manual which was reviewed by the assessors and
will describe the operations of the firm once it is fully compliant,
and the steps that are necessary to bring the firm into compliance.

The Operations and Policies Manual outlines the platform operator's
proposed operations and policies, as well as the steps it will take to
bring itself into compliance with applicable legal and regulatory
requirements. The manual serves as a comprehensive guide for the
platform operator's internal teams and stakeholders, providing
detailed information on the company's planned activities and the
measures it will implement to ensure compliance.

The manual covers various areas, including governance and staffing,
token admission, custody of VA, KYC, AML/CFT, market surveillance,
risk management, and cybersecurity. For each area, the manual
describes the current non-compliant practices and provides a clear
roadmap for achieving compliance.

In the section on governance and staffing, the manual includes an
organizational chart depicting the proposed management and governance
structure, business and operational units, and key human resources. It
outlines the steps the company will take to ensure the adequacy and
appropriateness of its corporate governance and staff resources,
considering the specific nature of VA trading activities. The
manual also describes the process for assessing personnel competency
and confirming that members of senior management have the relevant
industry experience, qualifications, technical expertise, and know-how
for their respective roles.

Regarding token admission, the manual details the establishment of a
transparent, fair, and properly documented token admission and review
committee. It outlines the criteria for admitting, halting,
suspending, and withdrawing VA's, ensuring compliance with relevant
regulatory requirements. The manual also describes the mechanisms for
ongoing monitoring of admitted VA's and regular reporting, ensuring
transparency and accountability.

In the section on custody of VA, the manual provides a comprehensive
explanation of the proposed wallet structure and systems, wallet
management policies, and governance procedures. It outlines the
operational flow of VA transfer between different wallets and confirms
that client assets will be protected in a manner comparable to
traditional financial institutions. The manual describes the specific
controls and mechanisms in place to ensure the secure storage of
client VA, including the use of cold storage, access authorization and
validation processes, and robust private key management procedures.

The manual also addresses KYC and AML/CFT measures, detailing the
proposed policies and procedures for customer due diligence, ongoing
monitoring, and compliance with regulatory requirements. It outlines
the steps the company will take to assess client knowledge of VA's,
determine risk tolerance levels, and set reasonable exposure
limits. The manual emphasizes the importance of implementing effective
screening methods, risk assessments, and controls to mitigate money
laundering and terrorist financing risks.

In the sections on market surveillance, risk management, and
cybersecurity, the manual describes the company's plans to establish
policies, controls, and systems to identify, prevent, and report
market manipulative or abusive trading activities. It outlines the
proposed external surveillance system, its parameters, alerts, and
methodology for detecting such activities. The manual also highlights
the company's commitment to implementing robust risk management and
cybersecurity measures, including regular testing, capacity planning,
and contingency arrangements.

Throughout the manual, the platform operator emphasizes its dedication
to compliance with legal and regulatory requirements. It provides a
clear timeline for the implementation of proposed operations and
policies, ensuring that the company will bring itself into compliance
within specified timeframes. The manual serves as a comprehensive
reference document, guiding the platform operator's efforts to
establish a compliant and secure trading environment for virtual
assets.

\chapter{Expertise and experience of the assessors}
\section{Capacity statement of assessor one}


\chapter{Methodology of the assessement}
\section{Scope/areas of the assessment assessment}

This report is the External Assessor Report for \firmfullname.  

The key documents from the Securities Futures Commission which define
the parameters of the Phase 1 assessment are:

\begin{itemize}
  \item Appendix F of the SFC Public Consultation request issued
    2023 February 20
  \item Questionnaire VA-1 of the SFC Licensing Application
\end{itemize}

The scope of the assessment included the areas indicated in Appendix
F.  Each topic item for the assessment corresponds to a specific
section of the Policy and Operations manual that was developed in the
course of the assessment.

\begin{itemize}
\item Part A—Governance and staffing—Governance
\item Part B—Token admission—Development
\item Part C—Custody of virtual assets—Operations
\item Part D—Know your clients—Operations
\item Part E—Anti-money laundering and counter-financing of
  terrorism (AML/CFT)—Operations
\item Part F—Market surveillance—Operations
\item Part G—Risk management—Risk management
\item Part H—Cybersecurity—Infrastructure
\end{itemize}


\section{Limitation of the Assessment}
Because the exchange is a new startup, the assessment by the external
assessors was made regarding plans and proposals by the
exchange to bring the exchange in compliance with the
requirements and the expectations of the Securities and Futures
Commission of Hong Kong.

The external assessors have made their assessments based only on the
Operations and Policy Manual.  For Phase 1 review, the assessors have
not examined the actual situation of the firm.

The financial projections are intended only to describe the necessary
financial resources necessary to bring the company into compliance.
The external assessors have been asked to verify that the resources
are sufficient to bring the firm into compliance.  They have not been
asked to verify the business models to generate the necessary revenue.

\section{Approach to the Assessment}

Because the exchange is a small startup that will need substantial
injections of capital and effort to bring it into compliance with the
standards expected by the Securities and Futures Commission, the first
step was for the firm to create an operations and policies manual
indicating the intended operations of the firm once the firm moves to
a fully licensed framework.

The second step was for the firm to conduct an internal
self-assessment, which consisted of a gap analysis indicating gaps
between the current state of the firm and the intended outcome.

Once the firm has completed a self-assessment, the firm then invited a
team of external assessors to assess the self-assessment and then
using their expertise to come to an agreed action plan for upgrading
the firm's operations to a completely licensed regime.

The assessors can then certify:

\begin{itemize}
  \item That the contents of the Operations and Policy manual
    addresses all the concerns of the Securities and Futures
    Commission and that, in the opinion of the assessor, is sufficient
    for the company to receive a licence
  \item That the gap between the current state of the firm and the
    Operations and Policy manual can be addressed by the firm
\end{itemize}

The EAR Phase 1 report is then to be submitted to the Securities and
Futures Commission for as part of the licensing application review and
comment.  The firm will change its Operations and Policy
Manual and action plan in response to feedback from the SFC.

After the consultation and discussion and after which the SFC is
satisfied with the objectives of the firm, the SFC will issue an
approval in principle.  At which time, the firm will execute the
action items in the Phase 1 report and a team of external assessors
will evaluate the completion of the objectives for the Phase 2 report.

\subsection{Coverage of assessment}
The assessment is based off Appendix F Part 1 of the SFC Public
Consultation request issued 2023 February and assessors were asked to
verify that all the requested topics were addressed in the
Operations and Policy manual as follows:
\begin{itemize}
\item Part A – Governance and staffing:
  \begin{itemize}
\item (i) Organizational chart: See section 2.1
\item (ii) Adequacy of governance: See section 2.2
\item (iii) Personnel competency: See section 2.3
\item (iv) Qualified professionals: See section 2.4
\item (v) Key personnel and key-man risk: See section 2.5
\item (vi) Management understanding: See section 2.6
\end{itemize}
\item Part B – Token admission:
\begin{itemize}
\item (i) Token admission committee: See section 4.1
\item (ii) Admission policies and procedures: See section 4.2
\end{itemize}

\item Part C – Custody of VA:
\begin{itemize}
\item (i) Wallet structure and systems: See section 5.3
\item (ii) Protection of client assets: See section 5.3.10
\item (iii) Private key management: See section 5.3.10
\item (iv) Storage assessment and testing: See section 5.3
\item (v) Deposit and withdrawal handling: See section 5.3.14/15
\item (vi) Reconciliations: See section 5.3.7
\end{itemize}

Part D – Know-your-clients (KYC):
\begin{itemize}
\item (i) KYC policies and procedures: See section 5.4
\item (ii) Acceptable account opening approaches: See section 5.4
\item (iii) Pre-implementation assessment: See section 5.4
\item (iv) Effective KYC process: See section 5.4
\item (v) Assessment of client knowledge: See section 5.4
\item (vi) Risk tolerance assessment: See section 5.4
\item (vii) Setting exposure limits: See section 5.4
\end{itemize}

\item Part E – AML/CFT:
  \begin{itemize}
\item (i) AML/CFT policies and controls: See section 5.6
\item (ii) Compliance with requirements: See section 5.6
\item (iii) Risk-based approach: See section 5.6
\item (iv) Identification of suspicious activities: See section 5.6
\end{itemize}
\item Part F – Market surveillance:
  \begin{itemize}
\item (i) Market surveillance policies: See section 5.8
\item (ii) Types of monitored activities: See section 5.8
\item (iii) External surveillance system: See section 5.8
\item (iv) Testing of surveillance system: See section 5.8
\end{itemize}
\item Part G – Risk management:
    \begin{itemize}
\item (i) Risk policies and procedures: See section 6.1.2
\item (ii) Independent risk management: See section 6.1.2
\item (iii) Risk management controls: See section 6.1.2
\end{itemize}
\item Part H – Cybersecurity:
  \begin{itemize}
\item (i) Cybersecurity risks: See section 7.2
\item (ii) IT budget and inventory list: See section 7.1.1
\item (iii) Compliance and security of IT infrastructure: See section 7.1.1
\item (iv) System management policies: See section 7.3.3
\item (v) Allocation of qualified staff: See section 3.7
\item (vi) Third-party service provider due diligence: See section 3.10
\item (vii) System upgrades and maintenance SOP: See section 7.3.3
\item (viii) Testing and audit trail for system modifications: See section 5.10
\item (ix) Client notification of system outages: See section 7.3.3
\item (x) Cybersecurity controls for platform security: See section 7.2.4
\item (xi) Escalation of cybersecurity incidents: See section 5.10.3
\item (xii) Capacity planning and contingency management: See section 7.3.3
\item (xiii) Contingency planning for emergencies: See section 7.3.3
\item (xiv) Backup facility and contingency plan review: See section 7.3.3
\end{itemize}
\end{itemize}

\subsection{Prompt for reviewer comments}
Reviewers were given the following prompts in relation to their
assessments:

\begin{itemize}
\item  Please state your qualifications for conducting this review

\item Referring to the requested information for a Phase One report,
  please verify that the operations and policy manual has addressed
  all the enquiries required by the SFC.  You may include statements
  concerning the limitations of your assessment.

\item Please state if in your professional judgment the proposed
  structure and procedures if implemented are adequate for running a
  fully regulated exchange in Hong Kong

\item Please state if, in your professional judgment, the management
  of the exchange are capable of implementing the proposed structure

\item Please state any suggestions for improvement and identify
  weaknesses and areas of concern in the self-assessment.  We are
  always interested in constructive criticism and opportunities for
  improvement, and would greatly appreciate comments and suggestions.

\item Please state if your professional judgement the SFC should or
  should not issue a VASP licence based on the information that has
  been provided to you by the exchange
\end{itemize}

\chapter{Governance and staffing / Token admission—Parts A-B}
\subsection{Self-assessment}
This self assessment covers the areas of governance and staffing and
token admission (Parts A and B)

We believe that our governance and controls are insufficient
to receive a VASP licence from the SFC.  All of our operations are
currently handled by one person, and this is clearly insufficient to
run a properly regulated exchange.  However, one reason that we are
going through the licensing process is to demonstrate to ourselves and
the world that one person with a vision can operate and thrive in Hong
Kong, and we are hopeful that our interactions with the Securities
Futures Commission in Hong Kong will be positive and will demonstrate
the pro-innovation and pro-business attitude of the Hong Kong
government.

We are conducting our planning activities as publicly as possible to
show the world that Hong Kong is ready for business and that a startup
exchange can work with the regulators to create win-win situations.

We are anxious to submit the VASP licensing application as quickly as
possible, as we have investors that will provide the required capital
necessary for us to create a properly regulated exchange once our
exchange has submitted a licence.

We believe that the key to proper governance is to have the right
people and the right structure, and we have outlined our proposed
governance structure and growth and development plans in Chapters 2
and 3 of our operations manual.  We would like the external assessors
to review our plans for governance and token admission to see if these
conform with SFC expectations.

In particular, we are planning to insure the proper running of our
exchange by separating out the day to day running of our exchange into
operations, technology, and custody as well as to having a separation
of rules between operations and compliance.  We believe that this
structure will provide us with the firm foundations to perform
regulated functions while providing security and reliability to the
Hong Kong investing public.

In contrast to the technical sections of the assessment, the speed at
which we can implement the development plans in Chapters 2 and 3 of
the operations manual is highly dependent on our ability to raise
funds.  The development plans in Chapter 2 make the assumption of full
staffing, and if we are unable to achieve full staffing
then we will modify our proposed governance structures while insuring
separation of roles.

We note that although we have the infrastructure to admit new tokens,
that we have no intention of trading tokens apart from those which are
already commonly trading on liquid markets.  We understand that to
issue proprietary tokens or to create a market for illiquid tokens and
derivatives would require substantial investments in internal policy
and controls, and we will refrain from such activities until we have
sufficient capital to do so.

We would like the external assessor on governance and compliance to
review the sections of our operations manual on our governance and
compliance sections as well as our proposed budget, and we look forward
to constructive criticisms and insights from the assessors.

\subsection{External assessor comments}

\chapter{Operations, Risk management, and Infrastructure Part C-H}

\subsection{Self assessment}

This self assessment covers the areas of custody of virtual assets,
KYC, AML/CFT, market surveillance, risk management, and cybersecurity.

Our exchange is novel in that it builds on publicly available open
source software.  Our exchange is based on the open-source software
system OpenCEX created by Polygant a Dubai-based company.  In addition,
we are currently in discussions with local Hong Kong firms to provide
upgrades to OpenCEX to make it compliant with SFC requirements.
However, we do not intend to have a software-as-a-service model and
intend that all operations will be run in-house.

We are particularly impressed by the wallet management model that is
used by OpenCEX.  Most systems divide the wallet management into hot
wallets and cold wallets, whereas the wallet management model which we
have described in the software infrastructure section of our
operations manual.

Although, OpenCEX provides an excellent open-source foundation for
further growth.  We have identified the following action items which
we believe will be necessary for our exchange to meet the requirements
of the Securities Futures Commission.

As we believe that we should implement the following procedures
immediately.  We project that these items will be implemented by the
end of September 2023.

\begin{itemize}
\item Migration of cold wallets to hardware wallets with
  multi-signature.

\item Upgrade of network infrastructure—place servers behind a Cloudflare and
OpenVPN firewall.  Install snort

\item Set up a user telegram group

\item We are currently using GitHub for bug tracking and technical
  reporting.  We will use another system such as linear.app for
  incident reporting

\item Upgrade the system to require two-factor authentication—2FA is
  currently optional for user account (although mandatory for staff
  accounts) and we will need to upgrade the system to make
  some form of 2FA mandatory

\item Currently, backups are made on the cloud server.  We will create
  a system to ensure that a copy of the backups are copied offline at
  least daily
\end{itemize}

We have a second set of action items which we will undertake upon
receipt of funding.  These will be performed with the
increased staffing which we plan for our company.  We project that we
will be able to complete these before January 2024.

\begin{itemize}

\item Staff separation of roles - OpenCEX has the ability to limit
  staff access to certain areas as part of its administrative system.
  However, we will need to develop the specific staff roles and the
  permissions necessary.

\item Test coverage—We will need to run test coverage checks and
  insure completeness of regression tests.

\item Integration of OpenCEX system with Scorechain KYT and Subsum KYC
  systems

\item Create a monitoring and accounting system to monitor wallet
  contents and generate automatic reports

\item Generation of documentation and terms and conditions for users

\end{itemize}

There is a third set of items which we will need to perform
before we being the EAR2 evaluation.

\begin{itemize}
\item Need to perform penetration testing
\end{itemize}

We will also need to perform the following software upgrade before we
are able to fully serve both professional and retail clients on the
same platform
\begin{itemize}
\item Limit trading based on client type
\end{itemize}
The functionality of limiting trading and products based on client
type will only be required if we wish to service non-HK and HK as well
as retail and professional investors on the same platform.  We can use
the current system without classification provided we limit trading to
the minimal set of pairs which are available to Hong Kong retail clients.

Once we have completed the EAR2 evaluation, these are the action items
that we will have to engage on an ongoing basis.

\begin{itemize}
\item cyberspace technology review
\item continuity of operations review
\end{itemize}

We note that at this point we are not planning to issue any tokens nor
are we planning to trade any tokens apart from those which are highly
liquid and commonly tradeable.  We have no current plans to trade such
tokens and understand that trading such tokens will require a
substantial investment in market surveillance and internal controls

\subsection{External assessor comments}

\end{document}
