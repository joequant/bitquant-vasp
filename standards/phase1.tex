\documentclass[]{report}
\usepackage{hyperref}
\hypersetup{
    colorlinks=true,
    linkcolor=blue,
    filecolor=magenta,      
    urlcolor=cyan,
    pdftitle={Overleaf Example},
    pdfpagemode=FullScreen,
    }

\def\firmfullname{Example Exchange Limited}
\def\firmshortname{Example}
\def\firmtag{example}

\title{Phase 1 External Assessor Report}
\author{\firmfullname\\Copyright 2023 - Licensed under Creative
  Commons CC:BY License}
\usepackage{fullpage}
\begin{document}
\maketitle
\section{Terms of Reference}

This report is the External Assessor Report for \firmfullname.  

It was generated with the following methodology with the documents
referenced available on github see \href{https://github.com/joequant/bitquant\-vasp}{href1}

The key documents from the Securities Futures Commission which define
the parameters of the Phase 1 assessment are:

\begin{itemize}
  \item Appendix F of the SFC Public Consultation request issued
    2023 February 20 see
    \href{https://github.com/joequant/bitquant-vasp/blob/main/regulations/sfc\-consultation\-20230220.pdf}{href2}
  \item Questionaire VA-1 of the SFC Licensing Application see
    \href{https://github.com/joequant/bitquant-vasp/blob/main/regulations/Questionnaire\-VA\-1.txt}{href3}
\end{itemize}

The first step was for the firm to create a operations and technical
manual indicating the intended operations of the firm once the firm
moves to a fully licensed framework.

The chapter headings of the policy and operations manual are set up to
correspond to the subject headings of Appendix F.

\begin{itemize}
\item Part A - Governance and staffing - Governance
\item Part B - Token admission - Development
\item Part C - Custody of virtual assets - Operations
\item Part D - Know your clients - Operations
\item Part E - Anti-money laundering and counter-financing of
  terrorism (AML/CFT) - Operations
\item Part F - Market surveillance - Operations
\item Part G - Risk management - Risk management
\item Part H - Cybersecurity - Infrastructure
\end{itemize}

The second step was for the firm to conduct an internal
self-assessment which consisted of a gap analysis indicating gaps
between the current state of the firm and the intended outcome.

One the firm has completed a self-assessment, the firm then invited a
team of external assessors to assess the self-assessment and then
using their expertise to come to an agreed action plan for upgrading
the firms operations to a completely licensed system.

The assessors are then able to certify:

\begin{itemize}
  \item That the intended objectives of the firm are sufficient for
    the firm to be licensed
  \item That the gap between the current state of the firm and the
    intended objectives are agreed to by all parties
  \item That the action plan and items necessary to bring the firm
    into full compliance are reasonable and can be achieved
\end{itemize}

The EAR Phase 1 report is then to be submitted to the Securities and
Futures Commission for review and comment.  After the consultation and
discussion and after which the SFC is satisfied with the objectives of
the firm, the SFC will issue an approval in principle.  At which time,
the firm will execute the action items in the Phase 1 report and a
team of external assessors will evaluate the completion of the
objectives for the Phase 2 report.

\section{Governance and staffing / Token admission - Parts A-B}
\subsection{Self-assessment}
This self assessment covers the areas of governance and staffing and
token admission (Parts A and B)

We believe that our governance and controls are insufficient
to receive a VASP license from the SFC.  All of our operations are
currently handled by one person and this is clearly insufficient to
run a properly regulated exchange.  However, one reason that we are
going through the licensing process is to demonstrate to ourselves and
the world that one person with a vision can operate an thrive in Hong
Kong, and we are hopeful that our interactions with the Securities
Futures Commission in Hong Kong will be positive and will demonstrate
the pro-innovation and pro-business attitude of the Hong Kong
government.

We are conducting our activities as publicly as possible to show the
world that Hong Kong is ready for business and that a startup exchange
can work with the regulators to create win-win situations.

We are anxious to submit the VASP licensing application as quickly as
possible as we have investors that will provide the required capital
necessary for us to create a properly regulated exchange once our
exchange has submitted a license.  We believe that the key to proper
governance is to have the right people and the right structure, and we
have outlined our proposed governance structure in Chapter 2 of our
operations manual.

In particular, we are planning to insure the proper running of our
exchange by separating out the day to day running of our exchange into
operations, technology, and custody as well as to having a separation
of rules between operations and compliance.  We believe that this
structure will provide us with the firm foundations to perform
regulated functions while providing security and reliability to the
Hong Kong investing public.

We would like the external assessor on governance and compliance to
review the sections of our operations manual on our governance and
compliance sections as well as our proposed budget and we look forward
to constructive criticisms and insights from the assessors.

\subsection{External assessor comments}

\section{Operations, Risk management, and Infrastructure Part C-H}

\section{Self assessment}

This self assessment covers the areas of custody of virtual assets,
KYC, AML/CFT, market surveillance, risk management, and cybersecurity.

Our exchange is novel in that it builds on publically available open
source software.  Our exchange is based on the open source software
system OpenCEX created by Polygant a Dubai based company.  In addition
we are currently in discussions with local Hong Kong firms to provide
upgrades to OpenCEX to make it compliant with SFC requirements.
However, we do not intended to have a software-as-a-service model and
intend that all operations will be run in-house.

We are particularly impressed by the wallet management model that is
used by OpenCEX.  Most systems divide the wallet management into hot
wallets and cold wallets, whereas the wallet management model which we
have described in the software infrastructure section of our
operations manual.

Although OpenCEX provides an excellent open source foundation for
further growth.  We have identified the following action items which
we believe will be necessary for our exchange to meet the requirements
of the Securities Futures Commission.

\subsection{Immediate action items}

\begin{itemize}
\item Migration of cold wallets to hardware wallets with multisignature

\item Upgrade of network infrastructure - place servers behind a cloudflare and
openvpn firewall.  Install snort

\item Set up user login monitoring

\item Set up user telegram group
\end{itemize}

We a second set of action items which we will undertake upon receipt
of funding

\begin{itemize}

\item Staff separation of roles - OpenCEX has the ability to limit
  staff access to certain areas as part of its administrative system.
  However we will need to develop the specific staff roles and the
  permissions necessary.

\item Test coverage - Need to run test coverage checks and insure
  completeness of regression tests.

\end{itemize}

There are a third set of items which will we will need to perform
before we begin our EAR2 report.

\begin{itemize}
\item Need to perform penetration testing
\end{itemize}

There are functions which we will need to implement within the
OpenCEX platform
\begin{itemize}
\item Fiat systems
\item P2P internal payments
\end{itemize}


\begin{itemize}
\item cyberspace technology review
\item continuity of operations review
\end{itemize}

\section{Assessors}
\section{Capacity statement of assessor one}


\section{Analysis}

\section{Statement of assessor}
\section{Capacity statement of assessor one}


\end{document}
