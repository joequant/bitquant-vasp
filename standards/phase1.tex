\documentclass[]{report}
\usepackage{hyperref}
\hypersetup{
    colorlinks=true,
    linkcolor=blue,
    filecolor=magenta,      
    urlcolor=cyan,
    pdftitle={Overleaf Example},
    pdfpagemode=FullScreen,    
    }

\def\firmfullname{Example Exchange Limited}
\def\firmshortname{Example}
\def\firmtag{example}

\title{Phase 1 External Assessor Report}
\author{\firmfullname\\Copyright 2023 - Licensed under Creative
  Commons CC:BY License}
\usepackage{fullpage}
\begin{document}
\maketitle

\chapter{Executive Summary}
This report presents the findings of the first-phase assessment
conducted on the design effectiveness of a platform operator. The
assessment covers various areas, including governance and staffing,
token admission, custody of virtual assets (VA), know-your-clients
(KYC), anti-money laundering and counter-financing of terrorism
... (AML/CFT), market surveillance, risk management, and
cybersecurity.

The assessment report highlights that the platform operator is
currently non-compliant with applicable legal and regulatory
requirements in certain areas. However, the report acknowledges that
the company has developed a comprehensive plan to rectify these
non-compliance issues and ensure future compliance.

As part of the assessment, the firm as developed a comprehensive
policy and operations manual which was reviewed by the assessors and
will describe the operations of the firm once it is fully compliant,
and the steps that are necessary to bring the firm into compliance.

The Operations and Policies Manual outlines the platform operator's
proposed operations and policies, as well as the steps it will take to
bring itself into compliance with applicable legal and regulatory
requirements. The manual serves as a comprehensive guide for the
platform operator's internal teams and stakeholders, providing
detailed information on the company's planned activities and the
measures it will implement to ensure compliance.

The manual covers various areas, including governance and staffing,
token admission, custody of VA, KYC, AML/CFT, market surveillance,
risk management, and cybersecurity. For each area, the manual
describes the current non-compliant practices and provides a clear
roadmap for achieving compliance.

In the section on governance and staffing, the manual includes an
organizational chart depicting the proposed management and governance
structure, business and operational units, and key human resources. It
outlines the steps the company will take to ensure the adequacy and
appropriateness of its corporate governance and staff resources,
taking into account the specific nature of VA trading activities. The
manual also describes the process for assessing personnel competency
and confirming that members of senior management have the relevant
industry experience, qualifications, technical expertise, and know-how
for their respective roles.

Regarding token admission, the manual details the establishment of a
transparent, fair, and properly documented token admission and review
committee. It outlines the criteria for admitting, halting,
suspending, and withdrawing VAs, ensuring compliance with relevant
regulatory requirements. The manual also describes the mechanisms for
ongoing monitoring of admitted VAs and regular reporting to ensure
transparency and accountability.

In the section on custody of VA, the manual provides a comprehensive
explanation of the proposed wallet structure and systems, wallet
management policies, and governance procedures. It outlines the
operational flow of VA transfer between different wallets and confirms
that client assets will be protected in a manner comparable to
traditional financial institutions. The manual describes the specific
controls and mechanisms in place to ensure the secure storage of
client VA, including the use of cold storage, access authorization and
validation processes, and robust private key management procedures.

The manual also addresses KYC and AML/CFT measures, detailing the
proposed policies and procedures for customer due diligence, ongoing
monitoring, and compliance with regulatory requirements. It outlines
the steps the company will take to assess client knowledge of VAs,
determine risk tolerance levels, and set reasonable exposure
limits. The manual emphasizes the importance of implementing effective
screening methods, risk assessments, and controls to mitigate money
laundering and terrorist financing risks.

In the sections on market surveillance, risk management, and
cybersecurity, the manual describes the company's plans to establish
policies, controls, and systems to identify, prevent, and report
market manipulative or abusive trading activities. It outlines the
proposed external surveillance system, its parameters, alerts, and
methodology for detecting such activities. The manual also highlights
the company's commitment to implementing robust risk management and
cybersecurity measures, including regular testing, capacity planning,
and contingency arrangements.

Throughout the manual, the platform operator emphasizes its dedication
to compliance with legal and regulatory requirements. It provides a
clear timeline for the implementation of proposed operations and
policies, ensuring that the company will bring itself into compliance
within specified timeframes. The manual serves as a comprehensive
reference document, guiding the platform operator's efforts to
establish a compliant and secure trading environment for virtual
assets.

\chapter{Expertise and experience of the assessors}
\section{Capacity statement of assessor one}


\chapter{Methodology of the assessement}
\section{Scope/areas of the assessment assessment}

This report is the External Assessor Report for \firmfullname.  

The key documents from the Securities Futures Commission which define
the parameters of the Phase 1 assessment are:

\begin{itemize}
  \item Appendix F of the SFC Public Consultation request issued
    2023 February 20 see
    \href{https://github.com/joequant/bitquant-vasp/blob/main/regulations/sfc\-consultation\-20230220.pdf}{https://github.com/joequant/bitquant-vasp/blob/main/regulations/sfc\-consultation\-20230220.pdf}
  \item Questionaire VA-1 of the SFC Licensing Application see
    \href{https://github.com/joequant/bitquant-vasp/blob/main/regulations/Questionnaire\-VA\-1.txt}{https://github.com/joequant/bitquant-vasp/blob/main/regulations/Questionnaire\-VA\-1.txt}
\end{itemize}

The scope of the assessment included the areas indicated in Appendix
F.  Each topic item for the assessment correponds to a specific
section of the Policy and Operations manual that was developed in the
course of the assessment.

\begin{itemize}
\item Part A - Governance and staffing - Governance
\item Part B - Token admission - Development
\item Part C - Custody of virtual assets - Operations
\item Part D - Know your clients - Operations
\item Part E - Anti-money laundering and counter-financing of
  terrorism (AML/CFT) - Operations
\item Part F - Market surveillance - Operations
\item Part G - Risk management - Risk management
\item Part H - Cybersecurity - Infrastructure
\end{itemize}


\section{Limitation of the Assessment}
Because the exchange is a new startup, the assessment by the external
assessors were made in reference to plans and proposals by the
exchange in order to bring the exchange in compliance with the
requirements and the expectations of the Securities and Futures
Commission of Hong Kong.

The external assessors have make their assessments based only on the
Operations and Policy Manual.  For Phase 1 review, the assessors have
not examined the actual situation of the firm.

The financial projections are intended only to describe the necessary
financial resources necessary to bring the company into compliance.
The external assessors have been asked to verify that the resources
are sufficient to bring the firm into compliance.  They have not been
asked to verify the business models to generate the necessary revenue.

\section{Approach the the Assessment}

It was generated with the following methodology with the documents
referenced available on github see \href{https://github.com/joequant/bitquant\-vasp}{https://github.com/joequant/bitquant\-vasp}

Because the exchange is a small startup and will need substantial
injections of capital and effort to bring it into compliance with the
standards expected by the Securities and Futures Commission, the first
step was for the firm to an create a operations and policies manual
indicating the intended operations of the firm once the firm moves to
a fully licensed framework.

The second step was for the firm to conduct an internal
self-assessment which consisted of a gap analysis indicating gaps
between the current state of the firm and the intended outcome.

One the firm has completed a self-assessment, the firm then invited a
team of external assessors to assess the self-assessment and then
using their expertise to come to an agreed action plan for upgrading
the firms operations to a completely licensed regime.

The assessors are then able to certify:

\begin{itemize}
  \item That the contents of the Operations and Policy manual
    addresses all of the concerns of the Securities and Futures
    Commission and that in the opinion of the assessor is sufficient
    for the company to receive a license
  \item That the gap between the current state of the firm and the
    Operations and Policy manual can be addressed by the firm
\end{itemize}

The EAR Phase 1 report is then to be submitted to the Securities and
Futures Commission for as part of the licensing application review and
comment.  The firm will make changes to its Operations and Policy
Manual and action plan in response to feedback from the SFC.

After the consultation and discussion and after which the SFC is
satisfied with the objectives of the firm, the SFC will issue an
approval in principle.  At which time, the firm will execute the
action items in the Phase 1 report and a team of external assessors
will evaluate the completion of the objectives for the Phase 2 report.

\chapter{Governance and staffing / Token admission - Parts A-B}
\subsection{Self-assessment}
This self assessment covers the areas of governance and staffing and
token admission (Parts A and B)

We believe that our governance and controls are insufficient
to receive a VASP license from the SFC.  All of our operations are
currently handled by one person and this is clearly insufficient to
run a properly regulated exchange.  However, one reason that we are
going through the licensing process is to demonstrate to ourselves and
the world that one person with a vision can operate an thrive in Hong
Kong, and we are hopeful that our interactions with the Securities
Futures Commission in Hong Kong will be positive and will demonstrate
the pro-innovation and pro-business attitude of the Hong Kong
government.

We are conducting our planning activities as publicly as possible to
show the world that Hong Kong is ready for business and that a startup
exchange can work with the regulators to create win-win situations.

We are anxious to submit the VASP licensing application as quickly as
possible as we have investors that will provide the required capital
necessary for us to create a properly regulated exchange once our
exchange has submitted a license.

We believe that the key to proper governance is to have the right
people and the right structure, and we have outlined our proposed
governance structure and growth and development plans in Chapters 2
and 3 of our operations manual.  We would like the external assessors
to review our plans for governance and token admission to see if these
conform with SFC expectations.

In particular, we are planning to insure the proper running of our
exchange by separating out the day to day running of our exchange into
operations, technology, and custody as well as to having a separation
of rules between operations and compliance.  We believe that this
structure will provide us with the firm foundations to perform
regulated functions while providing security and reliability to the
Hong Kong investing public.

We would like the external assessor on governance and compliance to
review the sections of our operations manual on our governance and
compliance sections as well as our proposed budget and we look forward
to constructive criticisms and insights from the assessors.

\subsection{External assessor comments}

\chapter{Operations, Risk management, and Infrastructure Part C-H}

\section{Self assessment}

This self assessment covers the areas of custody of virtual assets,
KYC, AML/CFT, market surveillance, risk management, and cybersecurity.

Our exchange is novel in that it builds on publically available open
source software.  Our exchange is based on the open source software
system OpenCEX created by Polygant a Dubai based company.  In addition
we are currently in discussions with local Hong Kong firms to provide
upgrades to OpenCEX to make it compliant with SFC requirements.
However, we do not intended to have a software-as-a-service model and
intend that all operations will be run in-house.

We are particularly impressed by the wallet management model that is
used by OpenCEX.  Most systems divide the wallet management into hot
wallets and cold wallets, whereas the wallet management model which we
have described in the software infrastructure section of our
operations manual.

Although OpenCEX provides an excellent open source foundation for
further growth.  We have identified the following action items which
we believe will be necessary for our exchange to meet the requirements
of the Securities Futures Commission.

\subsection{Immediate action items}

\begin{itemize}
\item Migration of cold wallets to hardware wallets with multisignature

\item Upgrade of network infrastructure - place servers behind a cloudflare and
openvpn firewall.  Install snort

\item Set up user login monitoring

\item Set up user telegram group
\end{itemize}

We a second set of action items which we will undertake upon receipt
of funding

\begin{itemize}

\item Staff separation of roles - OpenCEX has the ability to limit
  staff access to certain areas as part of its administrative system.
  However we will need to develop the specific staff roles and the
  permissions necessary.

\item Test coverage - Need to run test coverage checks and insure
  completeness of regression tests.

\end{itemize}

There are a third set of items which will we will need to perform
before we begin our EAR2 report.

\begin{itemize}
\item Need to perform penetration testing
\end{itemize}

There are functions which we will need to implement within the
OpenCEX platform
\begin{itemize}
\item Fiat systems
\item P2P internal payments
\end{itemize}


\begin{itemize}
\item cyberspace technology review
\item continuity of operations review
\end{itemize}


\end{document}
