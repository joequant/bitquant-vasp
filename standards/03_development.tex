\chapter{Development and growth}

\section{Token admission policy}
\crossref{VA-1: Question 13a}
\crossref{Appendix F1: Part B}
\status{To be created}

Tokens will be added onto the exchange only upon the recommendation of
an independent token admission board.  As we have no current plans to
list new tokens, we have not yet created such a listing committee.

The token admission committee will consist of the following members
who are appointed by and responsible to the board of directors, and
will exclude any persons with a direct commercial interest in the
listing of the token.

\begin{itemize}
  \item one member with regulatory and compliance background
  \item one member with background in decentralized finance and
    financial services
  \item one member with background in traditional finance
  \item one member with technical/coding background
  \item one member with consumer production background
\end{itemize}

The token admission committee will have a executive director and a
social media coordinator.

We expect that the members will be persons who are either retired from
distinguished careers or persons with academic backgrounds.  The names
of the members of the token review board are to be made public, and we
intend to have a public discussion board by which members of the
public can discuss which tokens and propose the addition of new
tokens.

One thing that we will try to do is to create open processes by which
our exchange is directly engaged with the investing community through
forums such as social media.

\subsection{Risk based token policy}
Each token subject to trading will be assigned a risk based assessment
by the token admission committee and the rationale for the assessment
shall be made public.  The risk based policy will be used to determine
the internal policies regarding trading of the token.

\begin{table}[htbp]
\centering
\begin{tabularx}{\textwidth}{|X|X|X|X|}
\hline
& \textbf{Category A tokens (Green)} & \textbf{Category B tokens (Yellow)} &
\textbf{Category C tokens (Red)} \\
\hline
Availablity & Token is widely available on other exchanges & Exchange
volume is a substantial volume of token & Exchange maintains some
level of exclusivity as to token trading \\
\hline
Market influence & Trades in the exchange will not effect
liquidity or pricing of the token & Trades in the exchange will
have limited liquidity or pricing of the token & Trades in the
exchange may substantially impact liquidity and pricing of the token \\
\hline
Market disruption & Disruption of trading will not affect trading and
liquidity of token & Disruption of trading will have limited impact on
trading and liquidity of token & Disruption of trading will have
limited impact on trading and liquidity of token  \\
\hline
Conflicts of interest & Exchange and related persons has no connection
with the token issuers and receives no special benefit / loss from price of
token and no access to insider information & Exchange and related persons has interactions with the token
issuers not available to general public and are receives some special
benefit / loss from price of token but no insider information & Exchange and related persons has
substantial interaction with issuers and potential or actual access to
insider information\\
\hline
Examples & BTC, ETH, BNB, TRX, USDT & hypothetical derivative with BTC
or ETH as price basis & ICO or securities token issued by exchange \\
\hline
\end{tabularx}
\caption{Policy classification of tokens}
\end{table}

All of the the products which are currently planned to be issued by
the exchange are Category A products, and the exchange will issue
public notices available on its website explaining the rationale for
the the classification.  Any public notification regarding trading of
tokens will make clear that these categories exist only for internal
compliance purposes and are not an endorsement or recommendation for
the token.

In the event, that the exchange will trade a Category B or Category C
token, the exchange management with the approval of the token issuance
committee will place appropriate trading restrictions on internal
trading.


\subsection{Due diligence}
\crossref{VA-1: Question 13b}
\status{Implementation in progress}

We expect that any tokens that we are listing will have already been
listed on a decentralized finance exchange.  Typically, a token issuer
will issue information in the form of a white paper, and we will
examine the white paper and public documents issued by the token
issuer, and gauge the sentiment of the investing public.

In some cases, we may contact the token issuer directly for
information.  However, in the interests of transparency wish to avoid
as much as possible the use of non-public information, and if we are
in a situation where we do not have sufficient information to list the
token, we would prefer that the issuer supplement their white papers
and public documents rather than engage in non-public discussions with
the exchange.

\subsection{Highly liquid assets}
\crossref{VA-1: Question 13c}
\status{Fully implemented}

The decision to classify an asset as highly liquid will be made by the
operations director after consultation with staff.  We do not believe
that this will be a controversial decision and if there is any doubt
as to whether or not an asset is a highly liquid large cap asset then
the determination is that it is not likely to be such.

In determining whether an asset is highly-liquid large-cap we will
use industry standard valuation boards including
\begin{itemize}
\item CoinMarketCap
\item CoinGecko
\item CryptoCompare
\item CoinCap
\item LiveCoinWatch
\item Coinlib
\end{itemize}

\subsection{Suspension and withdrawal from trading}
\crossref{VA-1: Question 13d}
\status{Fully implemented}
We believe that the most likely situation under which an asset will be
withdrawn from trading are situations in which there is some adverse
market disruption.  In this case the decision to suspend an asset from
trading will be done by the operations manager who will report the
decision to the managing director.

We note that because there are multiple exchanges, the decision by one
exchange to stop trading will not result in the removal of a market
for tokens, and hence in a particular exchange, the decision to stop
trading can be done by a single person and the existence of other
exchanges will provide check and balance in the system.

In determining whether a token is to be suspended from trading, the
operations manager would need to be guided by balancing the necessity
of maintaining a liquid market for a token versus issues of market
manipulation, fraud, and consumer protection.


\subsection{Anomalous incidents}
\crossref{VA-1: Question 14}
\status{Fully implemented}
In situations where there is adverse information or anomalous trading,
time and speed are of the essence, and we will set up an emergency
meeting by the operations manager and the managing director to deal
with this situation.  The principle that we will use is that we will
allow continued trading unless we believe that continuing trading will
result in fraud, undue enrichment, or market manipulation.

In cases where the trading is suspended, we will allow for client
withdrawals of tokens unless we believe that there is some underlying
issue with the blockchain which would allow for the furtherence of
fraud.

One issue that we will have to deal with is rules for coordinating our
activities with those of other exchanges.  In case of severe market
disruption, there will be a need for coordinated exchange action,
however such communications may lead to anti-competitive behavior.  We
would appreciate a mechanism by which exchanges can communicate with
each other and with regulators in cases of market disruption.

\subsection{Compensation for listed virtual assets}
\crossref{VA-1: Question 15}

We are not compensated for listed tokens and do not intend to accept
listing fees.  Because we are not charging listing fees, we cannot
support a heavyweight listing process, and so we would encourage
tokens that wish to be listed to present the information that we would
need to list in a public and transparent fashion.

\subsection{Current tokens traded}
\crossref{VA-1: Question 14}
\status{Fully implemented}
Right now, all of the tokens we trade are highly liquid large-cap
virtual assets.

\subsection{Airdrops and hard forks}
\crossref{VA-1: Question 16 and VA-1: Question 33}
\status{Terms and conditions to be written}
We will make very clear to our clients that they will not receive any
new tokens from airdrops and hard forks, and that any new tokens
created in company wallets will be credited to the company.

In case of an impending airdrop and hard fork we will inform our
clients of our policies and then facilitate their withdrawal of tokens
into their accounts so that they can take advantage of airdrops and
hard forks.



