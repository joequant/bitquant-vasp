\section{Governance of \firmshortname}
\crossref{VA-1: Question 2}
\crossref{Appendix F: Part A(i)-(vi)}
\crossref{Appendix F1: Part H(ii)-(vi)}
\crossref{VA-2: General controls 2.1}
\status{The full structure will be implemented once we have funding to
  hire staff to fill all of the roles}

The governance of \firmshortname is governed by the principle of
separation of roles.

\begin{figure}
\begin{forest}
  forked edges,
  for tree={
    draw,
    align=center
  }
  [Shareholders
  [Board
    [Managing Director
      [CTO
        [Technology staff]
      ]
      [Chief Custodian
        [Custodian staff]
      ]
      [\shortstack{Chief\\Operations\\Officer}
        [Operations staff]
        [Account management]
      ]
      [\shortstack{Chief\\Administrative\\Officer}
        [Finance]
        [Compliance and legal]
      ]
    ]
  ]
  ]
\end{forest}
\end{figure}

\section{Remote work policy}
We will allow operations staff, technology, and persons involves in
compliance and legal staff to work from locations outside of Hong
Kong.  We will not allow remote work for any persons involved in asset
custody, and people in the custody department must be physically
present in Hong Kong and must conduct their activities in our office
location.

\section{Part-time work}
We will allow staff to engage in work outside the firm provided
that they declare any outside interests and that these do not conflict
with the business of the firm.

In the case of wallet management, we will prefer a situation in which we
hire part-time additional resources to countersign wallet transaction
rather than allow one person to deal have transaction access.

\section{Cybersecurity risk management}
\crossref{VA-1: Question 51}
\status{Personnel not hired}

The responsibility for the cybersecurity is split between the chief
technology staff and the custody staff.  The technology staff is
responsible for the technology infrastructure including access to hot
wallets, while the custody staff will be responsible for the custody
of the cold wallets.  The reason for this dual role is to prevent
internal collusion and to make sure that no one person or group has
access to the entire system.

Our cybersecurity setup is designed to enforce a separation between
operations, custody, and technology.  The technology and systems used
to process orders will include open-source software based on OpenCEX
and Hummingbot, and is intentionally separate from the technologies
used for the custody function, which will include hardware wallets with
manual sign-offs.

Maintenance for all the systems will be done by the technology
department, which will not be allowed to routine access private keys
and the technology and will only be granted minimal access for
administrative purposes.

The order in which we plan to hire people are:

\begin{itemize}
\item Managing director
\item Finance and compliance officer
\item Technology officer 
\item Custody officer
\item Operator officer
\end{itemize}

In situations where there is a requirement for multi-signature wallet
access, the managing director will share a signature with the
department head.  As we build out the stuff, then managing director
will surrender his signing authority over to staff.

The governance structure that we have proposed is a base case
structure, assuming that we have met our funding and revenue targets.
In case we are unable to meet these targets, then to comply with the
operating capital guidelines in the VASP Guidelines 6.1 and the FRR we
will have to amend our structure to deal come up with a structure with
fewer staff, and we will inform the SFC and our EAR2 report reviewers
of the necessary amendments.

\subsection{Key man risks}
\crossref{Appendix F1: Part A(v)}
The managing director may appoint a deputy to exercise his duties in
his absence.  Without such an assignment, the financial and
compliance officer will exercise the duties of the managing director
in his absence.

In the case where a department chief is absent, the managing director
may exercise his duties or designate another person to exercise those
duties.

If it is necessary to revoke or reassign access to hardware wallets,
then the backups held by the legal counsel and the accounting firm
will be used for these activities.

In no situation should it be allowed that a single person can should
have access to the custody of cold wallets in a manner that would allow a
single person to gain access to client funds.

\section{Financial resources}
\subsection{Liquid Capital Requirements}
\crossref{VA-1: Question 6}
\crossref{VASP Guidelines 6.1 and 6.2}
\status{Capital fund raise upon license approval-in-principle}

The firm foresees no difficulty in complying with the minimum liquid
capital requirement of 5 million HKD.  The firm is in active
discussions with angel investors and venture capital and believes that
it will be able to raise the necessary liquid capital upon approval in
principle by the SFC. This investment will provide the firm with
necessary liquid capital before final issuance of the license.

Regarding the requirement of 12 months liquid capital in FRR
6.1, compliance with these guidelines will require that we are very
careful with hiring and operating costs to ensure that we
have a capital cushion that allows us to remain in operations.


\subsection{Auditor capability}
\crossref{VA-1: Question 8}
\status{Pending funding}
Pending funding by our investors, we have not sought auditors with
specialization in virtual assets.  We understand the importance of
doing so, and will engage the services of auditors that specialize in
virtual assets once we receive funding from our investors.

\section{Impact of business activities}
\crossref{VA-1: Question 6}

We project the following one time costs for the modifications of
business activities.  We do not project any changes in profit and less
for business activities

\begin{figure}[h]
  \centering
  \caption{Projected Impact.}
  \label{fig:financial_statement}
  
  \section*{Impact on business activities}
  
  \subsection*{One-time expense}
  \begin{tabular}{lr}
    Consultancy fees & HKD 50,000 \\
    Legal and professional fees & HKD 50,000 \\
    Software upgrade & HKD 50,000 \\
    Staff costs & HKD 0 \\
    \midrule
    Total Expenses & HKD 150,000 \\
  \end{tabular}
\end{figure}


\section{Projected financial statements}
\crossref{VA-1: Question 7}

Our projected balance sheet upon licensing is as follows.

\begin{figure}[h]
  \centering
  \caption{Financial Statement.}
  \label{fig:financial_statement}
  
  \section*{Balance Sheet}
  
  \subsection*{Assets}
  \begin{tabular}{lr}
    Cash and cash equivalents & HKD 6,000,000 \\
    Client cash assets (in segregated account) & HKD 3,000,000 \\
    Client virtual assets (in segregated accounts) & HKD 3,000,000 \\
    \midrule
    Total Assets & HKD 12,000,000 \\
  \end{tabular}
  
  \subsection*{Liabilities}
  \begin{tabular}{lr}
    Client cash liabilities & HKD 3,000,000 \\
    Client virtual assets & HKD 3,000,000 \\
    \midrule
    Total Liabilities & HKD 6,000,000 \\
  \end{tabular}
  
  \subsection*{Shareholder Equity}
  \begin{tabular}{lr}
    Total Shareholder Equity & HKD 6,000,000 \\
  \end{tabular}  
\end{figure}

Our projected monthly income / expenditure statements are as follows

\begin{figure}[h]
  \centering
  \caption{Financial Statement.}
  \label{fig:financial_statement}
  
  \section*{Monthly trading }
  
  \subsection*{Income}
  \begin{tabular}{lr}
    Fees and interest on margin loans & HKD 400,000 \\
    \midrule
    Total Income & HKD 400,000 \\
  \end{tabular}
  
  \subsection*{Expenses}
  \begin{tabular}{lr}
    Staff costs & HKD 300,000 \\
    Rent & HKD 30,000 \\
    Professional services & HKD 20,000 \\
    \midrule
    Total Expenses & HKD 350,000 \\
  \end{tabular}
\end{figure}

Under VASP Guideline 6.2, the operating expenses will require that we keep
HKD 4,800,000 in cash equivalents.
