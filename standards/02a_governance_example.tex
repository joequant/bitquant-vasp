\section{Governance of \firmshortname}
\crossref{VA-1: Question 2}
\status{The full structure will be implemented once we have funding to
  hire staff to fill all of the roles}

The governance of \firmshortname is governed by the principle of
separation of roles.

\begin{figure}
\begin{forest}
  forked edges,
  for tree={
    draw,
    align=center
  }
  [Shareholders
  [Board
    [Managing Director
      [CTO
        [Technology staff]
      ]
      [Chief Custodian
        [Custodian staff]
      ]
      [\shortstack{Chief\\Operations\\Officer}
        [Operations staff]
        [Account management]
      ]
      [\shortstack{Chief\\Administrative\\Officer}
        [Finance]
        [Compliance and legal]
      ]
    ]
  ]
  ]
\end{forest}
\end{figure}

\section{Cybersecurity risk management}
\crossref{VA-1: Question 51}
\status{Personnel not hired}

The responsibility for the cybersecurity is split between the chief
technology staff and the custody staff.  The technology staff is
responsible for the technology infrastructure include access to hot
wallets, while the custody staff will be responsible for the custody
of the cold wallets.  The reason for this dual role is to prevent
internal collusion and to make sure that no one person or group has
access to the entire system.

Our cybersecurity setup is designed to enforce a separation between
operations, custody, and technology.  The technology and systems used
to process orders will include open source software based on OpenCEX
and Hummingbot, and is intentionally separate from the technologies
used for the custody function which will include hardware wallets with
manual signoffs.

Maintainence for all of the systems will be done by the technology
department which will not be allowed to routine access private keys
and the technology and will only be granted minimal access for
administrative purposes.


\section{Financial resources}
\subsection{Liquid Capital Requirements}
\crossref{VA-1: Question 6}
\status{Capital fund raise upon license approval-in-principle}

The firm foresees no difficulty in complying with the minimum liquid
capital requirement of 5 million HKD.  The firm is in active
discussions with angel investors and venture capital and believes that
it will be able to raise the necessary liquid capital upon approval in
principle by the SFC, and this investment will provide the firm with
necessary liquid capital before final issuance of the license.

\subsection{Auditor capability}
\crossref{VA-1: Question 8}
\status{Pending funding}
Pending funding by our investors, we have not sought out auditors with
specialization in virtual asset.  We understand the importance in
doing so, and will engage the services of auditors that specialize in
virtual assets once we receive funding from our investors.

\section{Impact of business activities}
\crossref{VA-1: Question 6}

\begin{figure}[h]
  \centering
  \caption{Financial Statement}
  \label{fig:financial_statement}
  
  \section*{Income Statement}
  
  \subsection*{Revenue}
  \begin{tabular}{lr}
    Sales & \$10,000 \\
    Other Income & \$500 \\
    \midrule
    Total Revenue & \$10,500 \\
  \end{tabular}
  
  \subsection*{Expenses}
  \begin{tabular}{lr}
    Salaries & \$5,000 \\
    Rent & \$1,000 \\
    Utilities & \$500 \\
    \midrule
    Total Expenses & \$6,500 \\
  \end{tabular}
  
  \subsection*{Net Income}
  \[
  \text{Net Income} = \text{Total Revenue} - \text{Total Expenses} = \$10,500 - \$6,500 = \$4,000
  \]
\end{figure}


\section{Projected financial statements}
\crossref{VA-1: Question 7}

\begin{figure}[h]
  \centering
  \caption{Financial Statement}
  \label{fig:financial_statement}
  
  \section*{Income Statement}
  
  \subsection*{Revenue}
  \begin{tabular}{lr}
    Sales & \$10,000 \\
    Other Income & \$500 \\
    \midrule
    Total Revenue & \$10,500 \\
  \end{tabular}
  
  \subsection*{Expenses}
  \begin{tabular}{lr}
    Salaries & \$5,000 \\
    Rent & \$1,000 \\
    Utilities & \$500 \\
    \midrule
    Total Expenses & \$6,500 \\
  \end{tabular}
  
  \subsection*{Net Income}
  \[
  \text{Net Income} = \text{Total Revenue} - \text{Total Expenses} = \$10,500 - \$6,500 = \$4,000
  \]
\end{figure}

\begin{figure}[h]
  \centering
  \caption{Financial Statement}
  \label{fig:financial_statement}
  
  \section*{Balance Sheet}
  
  \subsection*{Assets}
  \begin{tabular}{lr}
    Cash & \$10,000 \\
    Accounts Receivable & \$5,000 \\
    Inventory & \$8,000 \\
    \midrule
    Total Assets & \$23,000 \\
  \end{tabular}
  
  \subsection*{Liabilities}
  \begin{tabular}{lr}
    Accounts Payable & \$3,000 \\
    Loans Payable & \$7,000 \\
    \midrule
    Total Liabilities & \$10,000 \\
  \end{tabular}
  
  \subsection*{Shareholder Equity}
  \begin{tabular}{lr}
    Common Stock & \$5,000 \\
    Retained Earnings & \$8,000 \\
    \midrule
    Total Shareholder Equity & \$13,000 \\
  \end{tabular}
  
\end{figure}
