The great challenge before us is to create standards which protect the
consumer while allowing for innovations and technology growth, and we
appreciate this opportunity to present to the Hong Kong Securities and
Futures Commission our vision and model for virtual asset regulation,
and how we intend to work with the regulators to create a safe and
virbrant virtual asset industry in Hong Kong.

The model for regulation that we are attempting to work with the
Securities and Futures Commission to create are akin to fire codes,
building codes, and food safety standards for restaurants.  One common
aspect of these types of health and safety regulations have been
designed to allow small businesses to comply with these standards, and
in fact creating safety codes in fact have the affect of promoting the
growth of small businesses and commerce.  The other common aspect of
health and safety regulation is that many of the standards have been
developed through difficult and painful experiences and mistakes
learned.

In developing these standards we have made use of the experience and
expertise of persons from the Hong Kong precious metals and gemstone
industry as well as persons in the money services industry.  In
traditional finance, physical security is not a primary concern
because the assets are not bearer assets.  A criminal who acquires by
force or fraud, bank statements or share certificates does not have
anything of value whereas virtual assets allow for a criminal with
physical possession of cryptocurrency to liquidate those assets in the
same way that they can liquidate gold or diamonds.  However, just at
it is possible to operate a gold or diamond trading business in Hong
Kong safely and profitably, we believe it is possible to create a
regulatory system by which one can run a virtual asset trading
platform business.

\section{The Regulatory model}

In creating a regulatory model one can use a rule based compliance
model or a principles based dialogue model.

The danger of a rules based compliance model is that we may end up in
a situation where there are hundreds of pages of rules which are
followed for the sake of following rules, and that these rules may end
up detached from the rational social objectives of these rules.  An
example of this situation occurred with the minibond crisis in Hong
Kong, where investment banks followed the letter of rules without
addressing the underlying rationale for those rules.

In addition, a regulatory model which is based on rules and compliance
runs the risk of being hijacked to create objectives which are often
contrary to the initial objectives or to any rational social purpose.

We believe that the SFC understands the limitations of a rule-based
compliance model, and is interested in a principles-based dialogue
model in which the regulators work with the industry in order to
implement standards and practices based on agreed principles and
objectives.  Where different stakeholders have different objectives,
this is a opportunity for dialogue and we consider the goal of the SFC
in undertaking licensing in during the transitional period to
encourage dialogue.


\section{Focus of regulation}

However the issue with regulation based on principles is that one may
then ask what the principles of the regulation are.


Means and ends / Relevance to the SFC / Ability of regulation to 
\begin{landscape}
\thispagestyle{empty} % Remove page number from the landscape page

\begin{table}[htbp]
  \centering
  \caption{Regulator Objectives}
  \begin{tabularx}{\linewidth}{|X|X|X|X|}
    \hline
    \textbf{Regulatory objective} & \textbf{Means versus ends} & \textbf{Role of SFC to regulate} & \textbf{Ability of SFC to regulate} \\
    \hline
    Investor asset protection & SFO Cap 571(4)(c) & Data 2 & Data 3 \\
    \hline
    National security & BL Article 1 & Data 4  & Data 5\\
    \hline
    Jobs and economic prosperity & Data 3 & Data 4  & Data 5\\
    \hline
    Social stability & Data 3 & Data 4  & Data 5\\
    \hline
    Technology innovation & BL Article 118 & Data 4  & Data 5\\
    \hline
    Market stability & Data 3 & Data 4  & Data 5\\
    \hline
    AML/KYC & Data 3 & Data 4  & Data 5\\
    \hline
    Investor suitability protection & SFO Cap 571(4)(a) & Data 4  & Data 5\\
  \end{tabularx}
\end{table}

\end{landscape}

In looking at the table of regulatory objectives, we believe that the
focus of virtual asset regulation should be investor asset protection,
and to ensure that investor assets are safely and securely stored by
the trading company.  We distinguish investor asset protection from
investor suitability protection.  Traditional securities regulation
has focused on ensuring products are suitable to the investor by
restricting access to products by investors.  Given the decentralized
nature of the virtual asset system, we believe that 

\section{Models for SFC regulation}

\subsection{Public regulated monopoly model (Exchange license)}
One possible model for SFC regulation is the model of the Hong Kong
Exchange.  Initally the Hong Kong market for stock exchanges consisted
of four exchanges which were merged in the 1980s and combined with the
Hong Kong Futures Exchange.  HKEx currently maintains an effective
monopoly on securities and futures trading in Hong Kong.

The creation of an regulated monopoly or oligopoly with high barriers
is also present in other industries.  For example, Hong Kong taxicabs
and hot dog stands in New York City are regulated for the purpose of
preventing excessive competition by creating barriers to entry and
ensuring health and safety and maintaining an orderly market.

However, applying the securities exchange model to virtual assets is
simply impossible for technological reasons.  HKEx is able to maintain
a monopoly on securities business because the HK Securities Clearing
Company Limited is the only company which can clear and settle
securities trades in Hong Kong using Hong Kong dollar.  By contrast,
virtual assets have a decentralized clearly and settlment model by
which there is no locus by which a regulated monopoly requirement can
be enforced.

\subsection{Technical standards model (Type 7)}

Another model for SFC regulation is the technical standards model
which is typified by regulation of automated trading systems.  In type
7 licensing SFC focuses on insuring that operators maintain high
standards of service and reliability and avoid from trades that create
conflict of interest.

Here to the ability of the SFC to regulate the market is limited by
technology.  Because Hong Kong securities are limited to clearing and
settlement on the HKEx, any automated trading system must be built on
top of these systems which requirements

1) Decentralized, which allows distributed clearing and settlement.
Second, for many highly traded tokens, list on multiple exchanges
making the reliability on one exchange less important.

\subsection{Brokerage model (Type 1)}
The most appropriate model which we believe for the exchanges is a
type 1 brokerage model. We note that Hong Kong has issed ****
brokerage licenses and that the barriers to entry to operate a new
brokerage are not excessively high.

We believe that in 

\subsection{Investment advisor and fund model (Type 4 and 9)}
Finally we wish to bring up type 4 and type 9 licensing.  Both type 4
and type 9 licensing are relatively ``light touch'' in comparison with
type 1 licensing as the barriers to entry for these industries are
relatively low, and it is in the interests of the SFC not to introduce
new regulatory barriers.  We 


\section{Business models}

\subsection{Internet platform model}

One business model which we was popular in the 2010s which we believe
is unsustainable is the internet platform model.  In the internet
platform model, a company is financed by venture capital to grab as
much market share as possible even if the company operates at a loss.
The belief is that once the company reaches a dominate market
position, it will then be able to operate at a profit and investors
will be able to recoup their investment.  This model has been highly
successful with internet companies such as Google and Facebook, and
has been the basis for models such as Uber or delivery services.

However, we believe that the internet platform model with respect to
exchanges is unsustainable.  We note that although there has been
market consolidation, that the nature of the coins makes it difficult
to create adequate returns based in exchanges.

\subsection{Professional services}

Another business model which is relevant to the discussion of is the
professional services model.  In this model there is a lucrative
generation of profits which creates a demand for professional
services.

However, we believe this model is unsustainable.  In order for a
professional services ecosystem to exist there must be some extra
surplus that is

Because it is now easy to move from one jurisdiction to another

\subsection{Restaurant ecosystem model}

\section{Safety standards models}

Given that we believe that primary goal of SFC regulation should be
investor asset protection, and given that we believe that the primary
philosophy of regulation should be a restaurant health and safety
model, we can describe how we have designed our systems.

Just as a ethical restaurant has both a business and a moral interest
in making sure that their customers are not ill from their food, we
consider it an essential part of our business do do the right thing,
based on past experience.

The design of these safety standards

* FTX
* 3 Arrow Capital
* Silicon Valley Bank
* Lehman Mini-bonds
* Hong Kong Merchantile Exchange
* hacks of bitcoin exchnages

Furthermore in designing a regulatory system we are particularly
concerned to avoid a situation where the regulation does not in fact
protect the consumer and becomes a hinderance to economic growth.  We
are particularly concerned at avoiding a "license raj" and to prevent
regulatory capture.  We are extremely concerned that regulations will
have the effect of creating extremely high barriers to entry that will
prevent small businesses from competing in the market and allowing for
only large companies with large amounts of capital to be regulated.
These large companies will then lobby for regulations whose effect is
to prevent other companies from competing.

The inevitable result of this situation is that when regulations are
focused at creating a cartel rather than consumer protection or
financial stability is an economic crisis when the regulations that
exist do not protect the consumer.

The main lesson that we believe to have been taught by FTX, 3 Arrow
Capital, and Lehman mini-bonds is the need to have clear ownership.  A
virtual asset trading company has three types of assets and three 
different stakeholders.

\begin{itemize}
\item assets owned by the client
\item assets owned by the business
\item assets owned by the persons running the business
\end{itemize}

In the FTX debacle, it was and still unclear which assets were owned
by clients, which assets were owned by FTX, which assets were owned by
Alameda Capital, and which assets were owned by Sam Bankman-Fried.
Similar confusing in asset ownership occured in the 3 Arrow Capital
situation and the Lehman Mini-bonds   The result of the consuming is
that the principals in the company then move client money into their
own pocket, and because this is other peoples money, they they have
the capital to undertake reckless and 

The design of the company is set up so that on day one there is a
clear division between client funds, business fund, and personal
assets, and that there are internal controls to prevent client assets
from being mixed with other forms of assets.  In the company design we
have set up our systems so that client assets are handled by a
dedicated custody team and that functionalities are separated so that
no one person has the internal ability to misappropriate client funds.

Once we have isolated the client assets from misappropriation and
placed them in a vault which is protected from misappropriation both
by governance and technology, we can then introduce other standards
which control access to the vault.  At this level we introduce
operational procedures so that AML/KYC and investor suitability
procedures are introduced, but as these measures are ``layered onto''
the basic structure, we are able to introduce AML/KYC and investor
suitability standards on top of the basic framework.


