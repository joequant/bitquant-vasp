\chapter{Mission and objectives}

\subsection{Mission statement}

The goal of Twofish Enterprises (Asia) Limited is to serve as
laboratory for financial technology.  Our philosophy and
organizational structure is modelled after an university research
group or a research and development arm of a major investment bank.
We intended to be the research and development center for the Hong
Kong virtual asset industry, focusing particularly on developing new
technology for small and medium enteprises in the greater bay area.

Our philosophy is that technological development does not merely
involve the development of better machines but to develop legal,
economic, and social structures to accommodate the new technology.  We
therefore intended to work closely with the Hong Kong government and
with financial regulators to help develop legal and regulatory systems
that promote the development of technology.

\subsection{Objectives}
We are operating a centralized exchange as a technology demonstration
platform, and therefore require licensing to operate these exchanges.
We are intenitionally keeping the size of the exchange small and
limited so that we can introduce and experiment with new technology
without causing risk to the wider financial community.

We do not expect to break even based on the fees from our exchange.
Moreover, we do not intended to charge licensing fees and all software
developed by us will be made available to users under an open source
license.  We are cooperating with other open source projects,
particularly OpenCEX and Hummingbot to build systems based on their
infrastructure.

Our revenue will come from technology consulting and using our
exchange to promote belt and road trade projects with Africa.  Because
these revenue streams are high margin we will not require a large
volume on our exchange to make our business a going concern.


\section{Regulatory agenda}

\subsection{Basic Law and one country two systems}


\subsection{Resolution and liquidation processes}

One failing of the current market structure is market disruption and
client difficulties when an exchange fails.  Failures of exchanges
have led to years of litigation and frozen client assets.  We believe
that the regulatory structure should create a system by which if an
exchange fails, that this matter can be dealt with very quickly and
efficiently with minimal client or customer impact.

\subsection{Collusion between exchanges}

Because the VASP will be listed on several different exchanges, there
will be a question of what communications will be considered
``self-regulation'' and which communications will be considered
anti-competitive behavior.  We would urge the SFC to set up mechanisms
involving ``chaperones'' where exchanges can coordinate but avoid
anti-competitive behavior.

\subsection{Self regulatory organizations}
We would encourage the creation of self-regulatory organizations in
order to maintain internal market discipline.  

\subsection{Policies toward non-HK investors}

We note that many of our clients are likely to be non-Hong Kong
residents and we would propose the following criterion.


\begin{itemize}
  \item The determination as to whether or not a person is a ``Hong
    Kong person'' for the purposes of the exchange will be determined
    by the documentation which they will provide concerning location
    of residency and identification.  Persons who are Hong Kong
    permanent residents but which present non-Hong Kong identification and
    location of residency will be classified as non-Hong Kong persons
    and will be subject the the regulation based on the location where
    they are identifed with respect to AML/KYC.
  \item Consumer protection and consumer information standards be made
    across all investors.  We note that many of our clients may come
    from areas with weak or non-existence consumer protections, and we
    would like to extend the benefits of Hong Kong rule of law and
    consumer protection standards to all of our clients worldwide.
  \item Token admission and trading will be determined by applicable
    local law
\end{itemize}

