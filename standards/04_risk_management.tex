\chapter{Risk Management}

\section{External risks}
\subsection{Market risks}
\subsection{Hacking risks}
\subsubsection{Mitigations}
\begin{itemize}
\item minimize the amounts of assets under custody
\item have mechanisms for rapid detection of suspicious activity
\item have the ability to have an emergency shutdown
  \item keep clear audit trails
  \end{itemize}
\subsection{Denial of service risks}
\section{Internal risks}
\subsection{Loss of key personnel}
\subsection{Rogue senior manager}
\subsection{Software failure}
\subsection{Loss of critical data}

Action item - insure that backups are done and perform quarterly tests
to insure that backups are recoverable

\section{Business risks}
\subsection{Critical vendor risk}
\subsection{Windup risks}

One key risk is what happens to the market and the clients in the
event of either a voluntary or involuntary windup.  It is possible
that business objectives will change, and in case of a windup, it is
the responsiblity of the firm to insure that this is done with
minimial disruption and that all creditors are quickly and promptly
paid.
