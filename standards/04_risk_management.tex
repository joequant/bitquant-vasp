\chapter{Risk Management}

\section{The risk management process}

The risk management process consists of five steps
\begin{itemize}
\item Identify the risk
\item Analyze the risk
\item Prioritize the risk
\item Mitigate the risk
\item Monitor the risk
\end{itemize}

This chapter of the policy management should be reviewed each quarter
by senior management in order to identify and prioritize each risk.
Analysis of risk should include not only impact on the firm but also
risks to financial stability.

\subsection{Windup and social impact risk}
Risks should include not only impact to the firm but also risk to
clients that result from a voluntary or involuntary liquidation.  In
case of a windup, the firm should be set up to be able to exit the
market with minimial disruption to clients and the overall market.

\section{External risks}

\subsection{Hacking risks}
\subsubsection{Mitigations}
\begin{itemize}
\item minimize the amounts of assets under custody
\item have mechanisms for rapid detection of suspicious activity
\item have the ability to have an emergency shutdown
  \item keep clear audit trails
\end{itemize}
\subsection{Market risks}

The main mechanism by which the firm will deal with market risk is not
to have mismatched liabilities and assets.  Any client funds held
should be held in the same tokens that the liabilities are created in
and should not be exchanged for other tokens.

\subsection{Denial of service risks}
\section{Internal risks}
\subsection{Loss of key personnel}
\subsection{Rogue senior manager}
\subsection{Software failure}
\subsection{Loss of critical data}

Action item - insure that backups are done and perform quarterly tests
to insure that backups are recoverable

\subsection{Growth risks}

\section{Business risks}
\subsection{Critical vendor risk}
\subsection{Windup risks}

One key risk is what happens to the market and the clients in the
event of either a voluntary or involuntary windup.  It is possible
that business objectives will change, and in case of a windup, it is
the responsiblity of the firm to insure that this is done with
minimial disruption and that all creditors are quickly and promptly
paid.
